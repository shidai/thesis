% vim:ts=4:sw=4
% Copyright (c) 2014 Casper Ti. Vector
% Public domain.

\begin{cabstract}

脉冲星类致密天体是宇宙中可直接观测的最致密的天体。它们不仅有丰富的多波段的观测现象,
还对于基础物理的研究有重要意义。除了可以用于检验引力理论,脉冲星类致密天体的结构
和状态方程直接反应了致密物质的物态、强相互作用以及QCD相变的本质。在我攻读博士学位
期间,我在北京大学和CSIRO Astronomy and Space Science展开了对于脉冲星类致密天体的
多波段的研究。

我的研究主要部分是关于射电脉冲星,尤其是射电脉冲星的高精度测时和脉冲星测时阵列(pulsar timing array, PTA)。
使用Parkes Pulsar Timing Array的数据,我们研究了24颗毫秒脉冲星的多波段的脉冲偏振
轮廓。为了理解多种与频率相关的效应对脉冲星测时精度的影响,我们开发了新的脉冲星
模拟和测时软件。我们还讨论了五百米口径球面望远镜(Five-hundred-meter Aperture Spherical Telescope)
对于未来脉冲星测时阵列研究的贡献。

为了寻找新的测量脉冲星质量的方法,我研究了由中子星和射电脉冲星导致的微引力透镜
事件的性质。我们讨论了未来的大型望远镜发现由射电脉冲星导致的astrometric microlensing
事件的可能性。我们的结果显示,通过astrometric microlensing现象,我们可以较精确
地测量脉冲星的质量。假设脉冲星的距离可以通过射电观测限制,那么脉冲星的质量可以
测量精确到约10\%。在X射线波段,我们研究了反常X射电脉冲星和软伽马射线重复包
的硬X射电谱。
	\pkuthssffaq
\end{cabstract}

\begin{eabstract}

Pulsar-like compact objects as the densest observable objects in the Universe, are not only
important in understanding diverse astrophysical phenomena, but also significant in fundamental
physics. Besides the physics of gravity, the answer to the question that whether pulsar-like
compact stars are neutron stars or quark stars would have profound implications on the physics
of condensed matter, the nature of strong interaction as well as QCD phase transition. During 
my PhD study at Peking University and CSIRO Astronomy and Space Science, I carried out multi-wavelength 
researches on pulsar-like compact objects. 

The main part of my study is focused on radio pulsars, especially on high precision timing of 
radio pulsars and pulsar timing arrays (PTAs). Using the Parkes Pulsar Timing Array data 
sets, we studied the multi-frequency polarization pulse profiles of 24 millisecond pulsars. We
developed new simulation and timing software packages to investigate various frequency-dependent 
effects on the precision of pulsar timing. We also studied how the Five-hundred-meter Aperture 
Spherical Telescope can contribute to the PTA studies. 

To find new ways to measure pulsar masses, I investigate properties of Galactic microlensing 
events caused by neutron stars and pulsars. We describe a possible study of astrometric microlensing 
events using radio pulsars that are likely to be discovered by future large telescopes. We show 
that such a study could lead to precise measurements of pulsar masses. For instance, if a pulsar 
distance could be constrained through radio observations, then its mass would be determined with 
a precision of $\sim$10\%. In the X-ray band, we studied the hard X-ray spectrum of AXPs and SGRs 
using XMM-Newton, Suzaku and INTEGRAL data. 

\end{eabstract}

