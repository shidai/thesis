% vim:ts=4:sw=4
% Copyright (c) 2014 Casper Ti. Vector
% Public domain.

\specialchap{结论}

我针对脉冲星类致密天体展开了多波段的研究,积累了多波段的观测、
数据处理和分析的经验。未来的脉冲星研究必将是联合多波段的观测进行的,
而未来多个地面和空间的、横跨多个波段的天文望远镜将有希望为我们
解开脉冲星类致密天体的本质的谜团,同时使我们可以通过脉冲星测时
探测和研究引力波辐射。

我的博士期间的工作可以总结如下:
\begin{itemize}
\item 射电波段的研究:

\begin{enumerate}
\item 我们研究了24颗毫秒脉冲星的多波段的脉冲偏振轮廓。长达六年的PPTA数据
使我们可以在三个波段给出高信噪比的脉冲轮廓,从而能够研究单个的脉冲成分以及
它们如何随频率演化。我们还发现了之前没有发现过的脉冲轮廓特征。对于我们的
样本中的多颗脉冲星,脉冲辐射几乎覆盖了整个脉冲周期。脉冲成分的宽度和脉冲
成分的间隔随频率的演化很复杂,一些情况下这些量随着频率的升高而增大,另一些
情况下则减小。偏振特性随频率的演化也极为复杂。我们发现脉冲轮廓的前导成分
和尾部成分相比主脉冲成分有比较高的线偏振度。我们使用三个频率的观测测量了
这些脉冲星的谱指数和法拉第旋转。对于大部分脉冲星,它们的谱符合幂率谱,而
偏振位置角也符合$\lambda^2$的关系。但是也有一部分脉冲星明显偏离这些关系。
我们还给出了每颗脉冲星的相位分离谱指数、线偏振度和法拉第旋转。我们发现
这些量在整个脉冲周期上有系统的变化,与脉冲轮廓的结构有关。
\item 我们研究了多种与频率相关的效应对高精度脉冲星测时的影响,特别是对于
宽波段的接收机系统。这些效应包括星际闪烁效应、色散延迟效应、脉冲轮廓随频
率的演化以及多普勒效应。我们开发了脉冲星观测的模拟软件,可以模拟星际闪烁
以及脉冲轮廓的演化。通过模拟,我们展示了这些效应对现有的数据以及未来的
宽波段观测的影响。为了消除这些效应,我们开发了新的脉冲星测时软件,进行
频率相关的测时,并且在测量脉冲到达时间的同时测量色散延迟。
\item 我们讨论了FAST在未来的PTA研究中扮演的角色。PTA的主要科学目标是探测
超低频的引力波、建立脉冲星时间标准以及改进太阳系的行星历书。而FAST将
发现大量新的毫秒脉冲星,并且大大地提高脉冲星测时的精度,于是将有可能
在这些方面扮演重要的角色。然而,在几年的时标上,FAST的脉冲星测时将受到
jitter noise的限制,而在更长的时标上脉冲星测时噪声将变得很重要。

\end{enumerate}

\item 光学波段的研究:

我们研究了银河系中由中子星和射电脉冲星导致的微引力透镜事件的性质。对于一个全天的photometric 
microlensing巡天,假设可以监测10$^{10}$颗背景天体,我们的估算显示,由约120000颗潜在可观测的射电脉
冲星导致的事件率为0.2\,yr$^{-1}$。考虑到astrometric microlensing的截面更大,因此我们期待对于
一个astrometric microlensing的巡天,假设相同的背景天体和可观测脉冲星数目,每年我们可以发现
几个事件。
%
我们的计算显示,对于由中子星导致的银河系photometric microlensing事件,持续的时标比之前的人们
的估算短,这主要是由于中子星比较大的自行。在银河系中心方向,对于时标约为15天的photometric 
microlensing事件,约有7\%是中子星导致的。对于更长时标的事件,这个比例快速下降。在远离银河系
中心的方向上,对于时标短于10天的事件,中子星贡献的比例可以高达40\%。这些结果是与之前的没有
考虑中子星分布的工作不同的,他们预测中子星将主要导致长时标的微引力透镜事件。
%
考虑到未来的大望远镜很可能发现有射电脉冲星导致的astrometric microlensing事件,我们研究了
通过这些事件测量脉冲星质量的精度。我们的计算显示,astrometric microlensing现象可以帮助
我们较准确地测量脉冲星的质量。如果脉冲星的距离可以通过射电观测独立测量,那么脉冲星的
质量可以被测量精确到约10\%。

\item X射线波段的研究:

多个反常X射电脉冲星和软伽马射线重复爆在10\,keV以上的硬X射线被探测到,但是目前仍不能理解
这些硬X射线辐射的起源。人们提出了多种模型来解释反常X射线脉冲星和软伽马射线重复爆的硬X射线
辐射的起源和物理机制,这些模型基于不同的对于这一类天体本质的理解,包括磁星模型和回落盘
系统模型。我们使用Suzaku和INTEGRAL卫星的数据研究了两颗反常X射线脉冲星(1RXS J170849‑400910,
1E 1547.0‑5408)和两颗软伽马射线重复爆(SGR 1806‑20,SGR 0501+4516)的软X射线和硬X射线辐射。
我们发现它们在软X射线和硬X射线辐射都能被整体康普顿化模型拟合(bulk-motion Comptonization)。
这说明在回落盘系统的模型下,反常X射线脉冲星和软伽马射线重复爆的X射线辐射可以被解释。我们
进一步对未来HXMT的观测进行了模拟,并且对于了不同的模型对硬X射线谱的预言。我们的模拟显示,
未来HXMT的观测很有希望帮助我们区分不同的辐射模型,从而帮助我们理解反常X射线脉冲星和软伽马射线
重复爆的本质。

\end{itemize}
\pkuthssffaq

