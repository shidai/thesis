% vim:ts=4:sw=4
% Copyright (c) 2014 Casper Ti. Vector
% Public domain.

\specialchap{序言}

脉冲星类致密天体是快速旋转的、高度磁化的中子星。它们最重要的特征
是从伽马射线波段到射电波段的多波段的脉冲辐射。由于中子星是宇宙中
最致密的天体之一,脉冲星类致密天体极为丰富的多波段现象使它们成为
研究致密物质的物态和引力理论的独一无二的天体物理实验室。而射电脉
冲星稳定的脉冲信号又使它们成为强有力的工具,被广泛用于研究星际介质、
探测引力波以及星际导航。

射电脉冲星于1967年被Jocelyn Bell-Burnell和Antony Hewish首次发现\supercite{hbp+68},
这一发现被授予了1974年的诺贝尔物理学奖。人们很快意识到,脉冲星
极短的自转周期意味着致密的物态,而脉冲信号来自强磁场中的辐射过程,
因而这类天体将向我们展现多种多样的极端现象。于是自发现以来人们一直
在使用多个大型射电望远镜以及利用多波段的观测搜寻脉冲星。目前为止我们
已经发现了2300多颗脉冲星类致密天体,其中大部分是射电脉冲星。对
脉冲星类致密天体的多波段研究也取得了丰富的成果。Russell Hulse和Joseph Taylor
在1975年发现了首个双中子星系统B1913$+$16\supercite{ht75}。
这个双星系统的轨道周期是7.75小时,并且由于引力波辐射损失轨道
能量,预计将在200兆年后并合。通过测量轨道周期由于引力波辐射损失
能量导致的缩减,Hulse和Taylor首次给出了引力波存在的证据,并因此
获得了1993年的诺贝尔物理学奖。首个脉冲星行星系统B1257$+$12在1990
年由Alexander Wolszczan和Dale Frail使用Arecibo望远镜发现\supercite{wf92}。
这个系统包含了两个地球质量的行星和一颗月亮质量的行星\supercite{wol94},
更重要的是这是第一次在太阳系外发现包含行星的系统。第一颗在球状
星团中的脉冲星由Andrew Lyne及其合作者发现\supercite{lbm+87}。在这
之后人们已经在球状星团中发现了几十颗脉冲星,这些发现使我们第一次
探测到了球状星团中的电离气体\supercite{fkl+01}。

除了射电脉冲星,使用多个X射线望远镜和伽马射线望远镜,人们还发现
了很多种脉冲星类致密天体,包括X射线脉冲星(X-ray pulsar)、中心致密
天体(compact central objects,CCO)、暗热中子星(dim thermal neutron star,DTN)、
反常X射线脉冲星(anomalous X-ray pulsar,AXP)、软伽马射线
重复爆(soft gamma-ray repeater,SGR)以及伽马射线脉冲星(gamma-ray pulsar)。
这些天体在X射电和伽马射线波段的辐射对于我们理解双星系统的演化、
脉冲星的辐射机制等有重要意义,而一些极端的爆发现象更是对于
中子星的物态提出了挑战\supercite{m08}。

我在博士期间对多种脉冲星类致密天体开展了多波段的研究。在射电
波段,我主要研究了毫秒脉冲星的多波段脉冲偏振轮廓、脉冲星的高
精度测时以及脉冲星测时阵列;在光学波段,我的工作主要是关于使
用微引力透镜的方法测量脉冲星的质量;在X射线波段,我研究了反常
X射电脉冲星和软伽马射线重复爆的硬X射线辐射。

\section{脉冲星的科学研究}

尽管脉冲星已经被发现四十多年了,并且人们开发和使用了大量的设备
和方法来对脉冲星类致密天体进行多波段研究,但是我们对于脉冲星的
理解还是非常有限的。我们仍然不清楚脉冲星的结构和内部的物态,
也不清楚脉冲星磁层的结构和辐射机制。我们仍然需要回答,脉冲星
是中子星还是夸克星?脉冲星的磁场和电场的结构是什么样的?磁层
中的等离子体密度是多少?射电辐射是如何产生的?高能辐射是从磁层
的什么位置产生的,以及如何产生的?目前我们还不知道上面任何一个
问题的确切答案,而这些问题的答案将直接为我们揭示致密物质的
物态和在极端条件下的等离子体物理。

另一方面,脉冲星极为稳定的自转以及辐射的脉冲信号使其成为研
究多种天体物理现象和基础物理问题的强大工具。虽然我们仍然不清楚
脉冲星内部的结构和物态,也不能完全理解多波段辐射的机制,
但是这并不影响我们使用脉冲星的稳定的周期性信号。特别是
通过射电脉冲星的高精度测时,我们可以研究双星系统的演化、
星际介质的性质、银河系的引力场和磁场,还可以检验引力理论
和探测引力波辐射。

要提取脉冲星脉冲信号中的信息,我们首先需要测量脉冲的
到达时间(pulse time of arrival,ToA),这个过程被成为
脉冲星测时。在射电波段,脉冲星的射电信号被射电望远镜
接收、放大,采样以及数字化。然后消除星际介质导致的色散
延迟,并在频率和时间上将观测叠加得到高信噪比的脉冲轮廓。
通过将脉冲轮廓与一个脉冲轮廓模板进行相关,我们得到脉冲
轮廓的相位移动,并最后转化为脉冲到达时间。这样得到的
脉冲到达时间是在观测者参考系下的,我们需要转换到太阳系
质心参考系,并且考虑太阳以及行星对脉冲到达时间的影响。
得到太阳系质心处的脉冲到达时间后,我们使用一个脉冲星
自转减慢、双星轨道和天体测量参数的模型来拟合到达时间,
然后将到达时间和最佳拟合模型之间的差别成为测时残差。
不正确或者不完备的脉冲星模型将导致测时残差中系统性的
结构,但是同时也给我们提供了研究脉冲星参数以及信号
传播过程的工具。例如我们通过脉冲星测时发现了脉冲星的
行星、测量了色散延迟随时间的变化等等。

多颗被精确测时的射电脉冲星可以组成脉冲星测时阵列(pulsar 
timing arrays,PTAs)。脉冲星测时阵列中的脉冲星的测时
残差的相关性可以帮助我们研究多种现象。脉冲星测时的
不稳定性、脉冲周期的跳变以及星际介质的变化将导致各个
脉冲星之间不相关的测时残差;地面时间标准的偏差将导致
各个脉冲星之间相同的测时残差表现;太阳系历书的误差
使各个脉冲星的测时残差与它们的位置有关;而由引力波
导致的测时残差的幅度和相位依赖于脉冲星和地球连线的
夹角。脉冲星测时阵列扩展了脉冲星测时的应用,发展了新的
探测引力波辐射的方法,也提供了研究太阳系的动力学、
脉冲星时间标准的途径。要实现脉冲星测时阵列的科学目标,
我们需要高精度的测时,目前只能在毫秒脉冲星上实现。
毫秒脉冲星有很短的自转周期,同时也有高度稳定的脉冲轮廓,
从而使测时的精度远高于常规脉冲星,达到几百纳秒的量级。
第一个脉冲星测时阵列,Parkes Pulsar Timing Array (PPTA)
于2004年使用Parkes望远镜建立\supercite{Manchester13}。
欧洲脉冲星测时阵列(European Pulsar Timing Array,EPTA)\supercite{fvr+10}
使用了Jodrell Bank、Effelsberg、Nancay以及Sardinia的望
远镜建立。北美纳赫兹引力波望远镜(North American Nanohertz Observatory of Gravitational Waves,
NANOGrav)\supercite{jfl+09}主要使用Green Bank和Arecibo
望远镜。这三个脉冲星测时阵列又一起组成了国际脉冲星测时
阵列(International Pulsar Timing Array,IPTA),总过
提供了约40颗毫秒脉冲星的高精度测时。国际脉冲星测时阵列
的数据最有希望在不远的将来首次探测到低频的引力波辐射。

\subsection{脉冲星的结构和物态}

脉冲星是在超新星爆发过程中产生的致密天体,假设质量是1.4倍太阳
质量,半径是10公里,那么脉冲星的密度将高达6.7$\times$10$^{14}$\,g/cm$^{-3}$,
甚至高于核物质密度2.7$\times$10$^{14}$\,g/cm$^{-3}$。在如此高的
密度下,物质的物态目前是不清楚的,极有可能出现夸克物质、玻色子
凝聚和奇异物质。Witten在1984年的工作中猜想\supercite{wit84},
奇异夸克物质可能是物质的基态,即在零压下有比铁更低的能量。根
据这一猜想,当物质的密度足够高时可能诱发从常规物质向奇异夸克物
质的相变,从而将整个星体变为奇异夸克星。这样的奇异夸克星将是
自束缚的,而不是像常规中子星那样引力束缚的。

传统的中子星模型认为脉冲星主要是由中子组成的。在中子星的表面
有一个固态的壳层,大概储存了中子星2\%的转动惯量。而在壳层之下,
是液态的内核。壳层被固定在液态内核上,而从壳层到内部,中子星
密度逐渐升高,跨越大概九个数量级。接近中子星表面,固态的壳层
主要由铁核和简并的电子组成,密度约为10$^{6}$\,g/cm$^{-3}$。
向中子星内部移动,密度逐渐升高直到质子和电子开始结合形成中子,
从而形成富中子的内壳层。在达到“中子滴”密度(4$\times$10$^{11}$\,g/cm$^{-3}$)
的半径之后(大约中子星表面之下几百米),中子所占的比例快速
提高。在密度超过2$\times$10$^{14}$\,g/cm$^{-3}$之后,中子星
将主要由超流态的中子组成,同时混合有大概5\%的超导的电子和质子。

随着人们意识到强子是由更基本的夸克组成的,以及夸克之间的色
相互作用的渐进自由特性,奇异夸克物质的稳定性及其天体物理后果
开始引起人们的关注\supercite{wit84}。1986年Haensel等人\supercite{hzs86}
和Alcock等人\supercite{afo86}等人基于自由夸克模型,首次计算
给出了奇异夸克星的结构并讨论了这样的天体的观测表现。然而,
我们知道脉冲星的温度往往只有几个keV,因此是属于低能量子
色动力学(QCD)的范畴,微扰QCD还有很大挑战。如果在几倍
核物质密度下夸克之间的色相互作用还比较强,致密物质中有可能
出现多夸克态,形成由u,d,s三味夸克组成的夸克集团。类似
于核子,夸克集团间可能还存在着剩余的短程排斥和远程吸引
相互作用。而当温度足够低时,夸克集团将像经典粒子那样
固化形成固态的夸克集团物质。

传统中子星模型和固态夸克集团星模型之间的差别主要体现在两个方面:
\begin{itemize}
\item 脉冲星的表面:传统中子星表面上的原子核和电子是被引力束缚的,
而固态夸克集团星表面的夸克是强相互作用束缚的,电子则是被强电磁力
束缚的。这种差别首先将导致不同的质量和半径的关系。对于引力束缚的
中子星,因为质量越大引力越强,半径随着质量的增加而减小;对于自束缚
的固态夸克集团星,表面密度非零,未达到极限质量之前半径随着质量的
增加而增加,特别是对于低质量夸克集团星,引力作用可以忽略,于是
质量正比与半径的三次方。其次,固态夸克集团星的强束缚表面意味着
可以存在裸的星体表面,而传统中子星表面通常被认为存在大气。这可以
自然地解释我们在无磁层活动的脉冲星的X射电波段的热辐射谱没有
观测到原子谱线。同时强束缚的表面还可以解释射电脉冲星的子脉冲
偏移现象以及超爱丁顿光度的爆发现象。
\item 脉冲星的结构:传统中子星模型和固态夸克集团星模型预言了
不同的星体内部密度和压强的关系,及状态方程。而状态方程的软硬
直接决定了脉冲星的极限质量。另一方面,固态夸克集团星整体都有
夸克集团构成,因而具有整体的刚性,而传统脉冲星由于内部的超流
结构不具备刚性。首先,与传统夸克星模型不同,固态夸克集团星内部
的夸克集团物质质量比较大因而是非相对论性的,同时有短程的排斥
相互作用,于是导致了更硬的状态方程。更硬的状态方程意味着更大的极
限质量,可以解释目前测量到的大于两个太阳质量的脉冲星。其次,
具有整体刚性的固态夸克集团星在自由转动和力矩作用下都有容易
产生进动,这可以自然地解释在一些脉冲星中观测到的进动。
\end{itemize}

尽管不同的脉冲星模型给出了不同的观测预言,但是现实的观测往往
是非常复杂的,很难提供确凿的证据来区分不同的模型。自束缚的星体
与引力束缚的星体往往有相似的半径、转动惯量以及中微子发射率和
透明度,而脉冲星的磁层和周围环境又极为复杂,因此很难通过光子
或者中微子观测,甚至射电脉冲星的测时来区分内部结构和物态。
最有希望的区分不同脉冲星模型和状态方程的途径是测量脉冲星的
质量和半径,特别是脉冲星极大和极小质量的测量。传统中子星模型
可以支持的最大质量是三倍太阳质量,这是基于广义相对论的结果
并且是有状态方程的软硬决定的\supercite{rr74}。当在高于核物质
密度时引入非核子自由度后,状态方程将变软,意味着更低的极限
质量,通常不高于两倍太阳质量。而对于固态夸克集团星,由于
夸克集团是非相对论的且有近程排斥,所以状态方程远比中子星硬。
在很宽的参数范围内,固态夸克集团星的极限质量可以高于两倍
太阳质量,并且可以高于三倍太阳质量\supercite{lgx13}。因此,
如果发现极大质量的脉冲星,那么我们将能极强地限制状态方程,
从而促进我们对致密物质物态的理解;而如果我们发现极小质量的
脉冲星,那么将说明脉冲星是自束缚而不是引力束缚的。

脉冲星的质量和半径的测量都是极有挑战性的。目前对脉冲星半径
的估测主要是通过脉冲星冷却过程中的热辐射、脉冲星在爆发过程
(包括热核爆炸、软伽马射线重复爆的巨爆发等)中的性质以及脉
冲星的周期跳变(glitch)现象等等。这些估测都还很不精确,很难
有效地限制脉冲星模型。脉冲星质量的测量要精确得多,这主要
得益于高精度的射电脉冲星测时。通过长期监测那些edge on的双星
系统中的射电脉冲星,我们可以精确地测量Shapiro延迟效应,从而
得到伴星,然后再根据双星系统参数测算出脉冲星质量。近年来
通过高精度的测时,人们已经测量到了两颗超过两倍太阳质量的
脉冲星\supercite{Anton,Demorest}。这两颗大质量脉冲星已经排除了很多传统
中子星模型和传统夸克星模型,并对于脉冲星的状态方程给出了
很强的限制\supercite{Ozel2010,Lai2011}。

但是目前对脉冲星质量的精确测量都是在双星系统中,孤立脉冲星
的质量测量还是一个挑战。而孤立脉冲星由于不在双星系统中,
可能有不同的演化过程,并且没有吸积的过程,因此可能有不同
的质量分布,更可能出现超大质量或者极小质量的脉冲星。一种
可能的测量孤立脉冲星的方法是通过微引力透镜效应\supercite{Dai,dsl+15},
我们将在后面的章节中具体讨论。

\subsection{脉冲星的磁层和辐射机制}

脉冲星是转动的、高度磁化的、超导的球体,在星体内部磁场,$\mathbf{B}$,
将导致一个电场,$(\mathbf{\Omega\times r)\times B}$。对于
一个导体,磁场导致的电场将被内部电子重新分布导致的电场,$\mathbf{E}$,
平衡,于是在星体内部任意一点,$\mathbf{r}$,电场力平衡给出
\begin{equation}
\mathbf{E}+\frac{1}{c}\mathbf{(\Omega\times r)\times B}=0.
\end{equation}
如果星体外部是真空,星体表面的电子将导致一个外部的四极电场,
\begin{equation}
\Phi(r,\theta)=\frac{B_{\rm{S}}\Omega R^{5}}{6cr^{3}}(3\cos^{2}{\theta}-1),
\end{equation}
其中$(r,\theta)$是以星体中心为原点的极坐标系,$B_{\rm{S}}$是
星体表面磁场强度,$R$是星体半径。在星体表面,平行于磁场
方向的电场强度是
\begin{equation}
E_{\parallel}=\frac{\mathbf{E\cdot B}}{B}\arrowvert_{r=R}=-\frac{\Omega B_{\rm{S}}R}{c}\cos^{3}{\theta}.
\end{equation}
使用脉冲星的典型参数进行估算,这个电场导致的电场力将超过
星体表面重力的十倍。因此在传统的中子星模型下,中子星表面的
粒子将被这个强电场拉出,于是在中子星周围形成等离子体。
脉冲星周围由磁场主导、等离子体填充的区域叫做脉冲星的磁层。

脉冲星磁层中的带点粒子在电场作用下也将重新分布,并且屏蔽
电场。根据电场力的平衡,在磁极处($r=R,\theta=0$),带点粒子
数密度,$n=\rho_{\rm{e}}/e$,可以计算得到
\begin{equation}
n_{\rm{GJ}}=\frac{\Omega B_{\rm{S}}}{2\pi ce}\simeq 7\times10^{10}\,\rm{cm^{-3}}(\frac{P}{s})^{-1/2}(\frac{\dot{P}}{10^{-15}})^{1/2}.
\end{equation}
这一电子密度是由Goldreich和Julian在假设脉冲星自转轴和磁轴
平行的情况下首次计算得到的\supercite{gj69},被称为Goldreich-Julian密度。

在脉冲星电磁场的作用下,磁层中的等离子体随着脉冲星共转。
然而,等离子体的共转速度不能超过光速,于是在某个极限距离
之后等离子体将不再随脉冲星共转。这个极限距离由等离子体的
共转速度等于光速定义,并且给出了一个假想的光速圆柱,半径为
\begin{equation}
R_{\rm{LC}}=\frac{c}{\Omega}\simeq 4.77\times10^{4}\,\rm{km}(\frac{P}{s}).
\end{equation}
光速圆柱将偶极磁力线分为了两类:1) 在光速圆柱内
闭合的磁力线(closed field lines);2) 开放的磁力线(open field lines)。
开放磁力线定义了脉冲星的极冠区域(polar cap)。极冠区域以
磁极为中心,它的边界($R,\theta_{\rm{p}}$)由最后
开放磁力线定义。最后开放磁力线是与光速圆柱相切的磁力线,
对于偶极磁场,$\sin^{2}{\theta}/r$是常数,于是我们得到
等式
\begin{equation}
\frac{\sin^{2}{\theta}}{r}=\frac{1}{R_{\rm{LC}}}=\frac{\sin^{2}{\theta_{\rm{p}}}}{R}.
\end{equation}
在脉冲星表面(取$R=10$\,km),极冠区的半径,$R_{\rm{p}}$,可以
估算为
\begin{equation}
R_{\rm{p}}\simeq R\sin{\theta{\rm{p}}}=150\,\rm{m}(\frac{P}{s})^{-1/2}.
\end{equation}

开放磁力线区域提供了射电脉冲辐射产生的通道。经典的射电辐射
模型认为磁层中的带点粒子以小集团的形式沿着磁力线运动,产生
曲率辐射。小集团的尺度不超过辐射波长的一半,其中的带点粒子
以相同的相位辐射,而辐射的能量是单个粒子辐射能量的粒子数目
的平方倍\supercite{kom70,rs75}。然而,曲率辐射的效率是比较
低的,而且人们的研究显示带电粒子的小集团很难在磁层中快速
形成,并且也无法在足够长的时间内保持集团形式\supercite{mel92}。
人们也提出了基于相对论性等离子体的辐射机制,这种模型需要
等离子体的不稳定性。除了等离子体的不稳定不能很快的扩大这个
问题外,这样的模型下等离子体湍动产生的能量也很难被传到
出去。人们提出了多种非线性机制尝试将等离子体湍动的能量
转化为某种模式的波,然后向外转移\supercite{mel92}。另外,
脉泽机制也被尝试用于解释脉冲星的射电辐射\supercite{mel89}。

到目前为止,脉冲星射电辐射的机制还不清楚。任何一种模型都
需要解释脉冲星射电辐射的高偏振度和相干性,同时还要理解在
跨越四个量级的自转周期和六个量级的磁场强度的多种情形下多种
观测特征,而且这样的辐射机制需要在很宽的频率范围内有效
($\sim$100\,MHz到100\,GHz)。

与射电辐射机制一样不清楚的是脉冲星辐射束的结构。在曲率辐射
机制下,带点粒子沿着开放磁力线运动,而辐射的能量随着磁力线
曲率的增大而增大。因此,我们期待最外开放磁力线上的辐射最强,
而往辐射束中心移动辐射强度逐渐降低,并在辐射束中心形成一个
空洞。当我们观测时,随着视线扫过辐射束我们将观测到一个有
双峰结构的脉冲轮廓,峰对应着辐射束的边沿,即是最外开放磁力线。
然而我们实际观测到的脉冲星脉冲轮廓是极为多样和复杂的,有的
脉冲星的脉冲轮廓只有一个成分,而一些有多个脉冲成分。为了
解释观测现象,人们提出了多种辐射束的结构模型。Backer 1976\supercite{bac76}
发展了cone模型,提出沿着磁轴有一个额外的辐射束,被称为
中心成分(core component)。对cone模型更进一步的改进是
猜测辐射束中有多个cone结构,中心是一个core成分\supercite{os76,os77,Rankin93,gks93},
于是多个脉冲成分的脉冲轮廓可以被解释。另一种模型认为脉冲星
的辐射束是由随即分布的离散的辐射区域(patch)填充的,并且
只有在开放磁力线区域的辐射斑是活跃的\supercite{Lyne88}。
混合了cone和patch的模型也被提出,Karastergiou \& Johnston (2007)\supercite{Kara07}
认为cone结构沿着靠近极冠边缘的磁力线分布在很宽的高度范围内,
而活跃的辐射区域是在cone结构之间随即分布的。在这种模型下,
年老脉冲星的脉冲轮廓比年轻脉冲星复杂可以得到解释。

人们尝试了各种途径来检验不同的辐射束模型。Cone模型预言
子脉冲和脉冲成分的相位随频率有微弱的演化,而patch模型
则预言没有演化\supercite{gk96,kg02}。Gil \& Krawczyk (1996)\supercite{gk96}
的工作发现观测支持cone模型,而Kijak \& Gil (2002)\supercite{kg02}
进一步的工作显示多个cone的模型更符合观测,包括脉冲轮廓
的opening angle的双峰分布现象以及单峰和双峰脉冲轮廓更大
的impact angle。为了研究脉冲星辐射束的二维图像,Han \& Manchester (2001)\supercite{Han01}
对87颗有多个脉冲成分的正常脉冲星进行分析,发现除了靠近
辐射束中心的强度有所增加,在辐射束的其他位置强度没有明显
的分布,从而支持了patch模型。在双星系统中发现的脉冲星的
进动现象使我们有机会扫描探测辐射束的不同部分,从而研究
它的结构。目前,人们已经对六颗有进动的脉冲星PSRs J1913$+$16,
J1534$+$12,J1141$-$6545,J1906$+$0746,J0737$-$3039A和
J0737$-$3039B进行了辐射束的研究\supercite{kra12},发现辐射束中没有明显
的cone结构的证据,但是看到辐射束中有部分区域被照亮,
这些结果明显支持patch模型。最近,Wang et al. (2014)\supercite{Wang14}
讨论了基于宽波段、相干辐射的fan beam模型,从不同的角度
解释脉冲轮廓的成分及其随频率的演化。

对于脉冲星辐射机制和磁层结构的研究直接与强磁场下的等离子体
物理相关,同时也有助于我们理解磁层当中的不稳定性,从而
促进脉冲星测试精度的提高。目前已有的脉冲星辐射和磁层模型
还远不能解释我们观测到的所有现象。我们需要更高精度、更高信、
噪比的观测,在后面的章节中我们将详细介绍与毫秒脉冲星的
脉冲偏振轮廓相关的工作。

\subsection{银河系的结构和星际介质的研究}

脉冲星在银河系中的分布范围很广,考虑到脉冲星比较的自行,
更是有可能分布在银河系的任何区域。于是结合射电脉冲星的
偏振脉冲信号,我们可以利用射电脉冲星研究银河系的结构和
星际介质的性质。

射电脉冲星的色散延迟直接反应的是自由电子在视线方向上的
积分柱密度。如果银河系的自由电子分布是已知的,那么我们
可以通过色散测量(dispersion measure,DM)估算脉冲星的
距离。实际上,银河系的自由电子分布是不清楚的,可是如果
脉冲星的距离可以通过其他方式独立的测量,比如通过视察和
中性氢吸收,那么我们可以使用这些脉冲星的色散测量来研究
自由电子的分布。对于毫秒脉冲星,由于测时的精度能达到
很高,使得我们甚至可以研究色散测量在较小空间尺度上的
变化。例如通过观测球状星团47 Tucanae中的16颗毫秒脉冲星,
Freire et al. (2001)\supercite{fkl+01}给出在大约5\,pc的
距离上,色散测量变化了大约2\%。这意味着在这个球状星
团中的电离气体的自由电子密度约为0.07\,cm$^{-3}$。

另一方面,脉冲星的射电脉冲辐射是高度线偏振的。于是
射电脉冲信号在银河系的磁场中传播将会有法拉第旋转效应,即
线偏振位置角会产生正比于视线方向的磁场强度和电子密度的
旋转。通过测量大量的射电脉冲星的法拉第旋转效应,我们
可以还原银河系的磁场的大尺度结构。Han et al. (2006)\supercite{hml+06}
给出了223颗脉冲星的法拉第旋转,并结合之前的测量,
展示了银河系旋臂中逆时针方向的磁场。

脉冲星的监测还能反应星际介质在较短时标和较小空间尺度
上的变化。对于一些在超新星遗迹中的脉冲星,当小团的电离
气体经过我们到脉冲星的视线方向时,色散延迟会发生变化。
例如,Vela脉冲星的色散测量在1970年到1985年之间的变化
率约是0.04\,cm$^{-3}$\,pc\,yr$^{-1}$\supercite{hhc85}。而
Crab脉冲星的色散测量在15年的时间里变化率约为0.02\,cm$^{-3}$\,pc\,yr$^{-1}$\supercite{lps88}。
有掩蚀的脉冲星双星系统中,脉冲星的色散测量也有可能
因为伴星电离包层的影响而变化,例如PSR B1259$-$63的
色散测量的变化在近星点附近能够达到约10\,cm$^{-3}$\,pc\supercite{wjm04}。
星际介质自身的湍动也可以造成脉冲星的色散测量随时间
的变化。使用PPTA的多波段数据,You et al. (2007)\supercite{yhc+07}
和Keith et al. (2013)\supercite{Keith13}给出了PPTA
毫秒脉冲星的色散测量在几年的时间上的变化。对这些
色散测量变化的研究能够促进我们对湍动过程的理解。

\subsection{引力理论的检验}

脉冲星的致密性使其成为理想的引力实验室,并且能提供
地面实验室不能实现的强引力场极限。利用脉冲星的高精度
测时,包含脉冲星的双星系统成为唯一的检验广义相对论
预言以及不同的引力理论的场所。

双中子星系统中的中子星几乎可以被当作质点,因此是最为
理想的检验强引力场极限下的引力理论的系统。目前发现的
双中子星系统的轨道周期都在2.4小时到18.8天之间,意味着
双星轨道长度都远大于中子星半径,于是没有中子星之间的
质量传输,也没有潮汐效应。最著名的例子是B1913$+$16,
通过长期的脉冲测时,发现双星的轨道半径由于引力波的辐射
每天减小1\,cm,从而间接地证明了引力波的存在。

使用双中子星系统检验引力理论主要是通过脉冲星测时测量
后开普勒参数(post-Keplerian parameter)。对于可以忽略
自转的质点,后开普勒参数可以用双星系统中两颗星的质量
以及开普勒参数来表达。如果测量了两个后开普勒参数,那么
一种引力理论将可以分别给出两颗星的质量。更广义地说,
如果测量了n个后开普勒参数,那么在两颗星的质量的参数
平面上将给出n条曲线,曲线的形状和位置是由引力理论决定
的。对于任何一个可行的引力理论,所有n条曲线应该交于一点,
于是给我们提供了n-2中对引力理论的测试。目前为止最好
的例子是J0737$-$3039这个系统,一颗周期为22.7毫秒的脉冲星
(称作A)绕一颗周期为2.8秒的脉冲星(称作B)运动。仅仅
经过12个月对A的测时监测就给出了五个后开普勒参数,而
由于B也是一颗射电脉冲星,测时可以给出A和B的质量比,从而
又提供了更强的限制。通过这个系统,广义相对论得到了
很强的检验,并且很好地符合观测。而在未来,经过更长
的监测,这个系统有望检验更多的理论预言。

与双子星系统类似,中子星和白矮星的双星系统中的两个
天体也可以近似看作质点。相比于双中子星系统,中子星
和白矮星的双星系统的广义相对论效应要弱得多,然后一些
开普勒参数还是可以测量的,并且可以检验引力理论的一些
方面。与广义相对论不同,一些引力理论(例如tensor scalar理论)
预言了与双星系统中两个天体的质量差强烈相关的效应。
尽管一部分这种效应可以在太阳系中检验,但是只能在
弱引力场极限下,而一些理论确实可以通过太阳系的检验\supercite{de96}。
一些双中子星系统可以在强引力场极限下给出与太阳系
检验相当的检验。对于中子星和白矮星双星系统,两个
天体的质量差很大,因此是最理想的在强引力极限下检验
上述效应的系统。另一方面,一些引力理论预言了偶极的
引力波辐射,而广义相对论预言引力波辐射最低是四极的。
偶极引力波辐射在中子星和白矮星系统中效应是最大的,
因为它们的质量差很大。目前对于偶极引力波辐射最强
的限制就是使用中子星和白矮星系统J1012$+$5307给出的\supercite{lcw+01}。
中子星和白矮星系统还可以用于检验强等效原理(strong 
equivalence principle)。广义相对论是满足这个原理
的,而很多引力理论则不满足。由于中子星和白矮星的
质量差很大,在银河系的引力场作用下,如果强等效
原理不满足,那么双星的轨道将受到影响。长周期、圆
轨道的中子星和白矮星系统可以在强引力场极限下检验
强等效原理\supercite{ds91}。

\subsection{探测低频引力波辐射}

Sazhin (1978)\supercite{saz78}首先讨论了由恒星质量的双星
系统和超大质量双黑洞系统辐射的引力波在脉冲星测时信号中
可能引起的效应,并且指出超大质量双黑洞系统辐射的引力波有
可能被脉冲星测时探测到。Detweiler (1979)\supercite{det79}
进一步发展了这一思想,首次利用脉冲星测时残差给出了随即背景
引力波辐射的上限。随后,Hellings \& Downs (1983)\supercite{hd83}
指出背景引力波辐射可以在不同脉冲星的测时信号之间引起
相关,并给出了相关性与脉冲星之间位置夹角的关系。在这些
工作的基础上,通过脉冲星测时阵列探测引力波辐射的想法
被提出\supercite{Foster90}。脉冲星测时阵列
的数据对引力波敏感的波长的上限由监测的总时间跨度决定,
而下限由观测的频率决定。因此脉冲星测时阵列对超长波长,
或者说超低频的引力波灵敏,频率范围为10$^{-9}$到10$^{-8}$\,Hz。
脉冲星测时阵列所敏感的频率范围正好与地面引力波探测器,
比如LIGO,的灵敏探测范围互补。

脉冲星测时阵列最有可能探测到的是宇宙中的背景引力波辐射。
背景引力波可能来自于宇宙学超弦\supercite{sbs12}、暴涨阶
段\supercite{zhao11}以及超大质量双黑洞的并合\supercite{svc08}。
由背景引力波辐射导致的脉冲星测时残差由一个特征的类似
红噪声的能谱\supercite{hobbs12b}:
\begin{equation}
P(f)=\frac{A^2}{12\pi^2}(\frac{f}{f_{\rm{1yr}}})^{2\alpha_{\rm{GW}}-2}.
\end{equation}
图\ref{limit}中的几条水平线给出了近些年的工作给出的引力波
辐射上限\supercite{jhv+06,vlj+11,dfg+13}。

随着人们研究的深入,我们对于引力波信号的预期也在变化,
在图\ref{limit}中这表现为阴影和矩形区域。最左边的边界
是在Jenet et al. (2006)\supercite{jhv+06}发表时的预期。
现在主要的变化是:1) $\alpha_{\rm{GW}}$的可能的范围明显变
宽了,而宇宙学超弦预计的引力波下限远低于已有数据给出
的限制;2) 暴涨可能贡献的引力波辐射强度是比较低的;3) 
由超大质量双黑洞并合导致的背景引力波比原来预计的更平。

早期的工作都假设可以探测的引力波是各向同性、随机的背景。
然而最近的研究表明,我们也可以探测由单个超大质量双黑洞
系统并合产生的引力波、双黑洞并合的记忆效应以及爆发事件。
近期的工作也对这些引力波源进行了限制\supercite{Wang15,Zhu14}

\begin{figure}
\centering
\includegraphics[width=7cm,angle=-90,trim=1cm 1cm 0cm 1cm]{limit.ps}
\caption{预言和由实际观测给出了背景引力波辐射的上限。图片来自Hobbs (2012)\supercite{hobbs12b}。}
\label{limit}
\end{figure}

\subsection{天体力学和天体测量}

为了得到脉冲星的本征脉冲到达时间,我们需要将望远镜记录
的脉冲到达时间换算到太阳系质心处。这一转换敏感地依赖于
脉冲星的位置。当测时精度比较高时,任何细微的脉冲星位置
的偏差都将导致测时残差中明显的结构。因此通过脉冲星的高精度
测时,我们可以精确地测量脉冲星的天体测量参数。对于常规
脉冲星,通过测时得到的脉冲星位置可以精确到几百毫角秒甚至
更低。而对于毫秒脉冲星,测时精度更高,位置的测量精度可以
高于毫角秒。例如对于最亮的毫秒脉冲星PSR J0437$-$4715,
位置的测量精度能达到微角秒的量级。对于在球状星团中的脉冲星,
精确的位置测量对于研究星团的物理性质有重要作用。

通过脉冲星测时得到的脉冲星位置是在黄道坐标系下的,而使用
其他方式测量的位置则在不同的坐标系下。比如,使用射电干涉
测量的脉冲星位置在赤道坐标系下,并且是于地球的自转以及
类星体的位置相关的。对比不同坐标系下脉冲星的位置可以帮助
我们确定不同坐标系的关系,这往往是和地球的自转轴相关的。
Madison et al. (2013)\supercite{Madison13}计算了International 
Celestial Reference Frame (ICRF)与脉冲星测时坐标系统的
转换,并且讨论了在未来这样的转化能如果通过更多的观测改进。

脉冲星的自行相比于银河系内的其他天体是比较大的\supercite{hobbs}。
一般认为脉冲星比较大的自行来自于超新星爆发时的不对称性导致
的kick速度以及超新星爆发之前的轨道速度。因此对于脉冲星自行
的研究将促进我们对于脉冲星初始速度分布以及超新星爆发的机制
的理解。

\section{脉冲星的应用研究}

脉冲星的研究也具有非常重要的应用价值,主要体现在两方面:1) 
脉冲星时间标准\supercite{hcm+12};2) 脉冲星导航\supercite{dhy+13}。
这些应用都依赖于高精度的脉冲星测时,因而与我们对于脉冲星的
结构和辐射机制、星际介质的性质以及脉冲星测时的算法的研究
密切相关。

\subsection{脉冲星时间标准}

地面时间(terrestrial time,TT)是由原子频率和原子钟给出的。
各个国家都有当地的原子钟时间标准,然后各国的原子钟时间被Bureau 
International desPoids et Mesures (BIPM)集合起来得到国际原子钟
时间(International Atomic Time,TAI)。国际原子钟时间是Coordinated 
Universal Time (UTC)的基础,然后作为标准时间信号和地面时间。

尽管国际原子钟时间是有很多的原子钟平均得到的,并且世界各国的
原子钟在不断的发展,变得越来越稳定,但是在几十年的时标上仍然
很难保证时间标准的稳定。Hobbs et al. (2012)\supercite{hcm+12}
中展示了国际原子钟时间自1994年以来相对于每年的修正时间标准的
变化,我们可以看到变化的幅度约为五微秒。因此,我们有必要建立
一个独立于原子钟时间的时间标准,并且是在较长的时间尺度上
稳定的。

脉冲星有极为稳定的自转周期,并且这样的稳定自转可以通过脉冲星
测时加以利用,因此成为理想的建立时间标准的工具。相比于原子钟,
脉冲星时间标准有几个独特的优势:
\begin{itemize}
\item 脉冲星是太阳系外的天体,因此提供了对地面时间标准独立
的检验;
\item 原子钟是基于微观量子物理的,而脉冲星是宏观的大质量天体;
\item 脉冲星时间标准可以存在极长的时间,远远长于任何的原子钟。
\end{itemize}

目前测时精度最高的是毫秒脉冲星,对于一个小时的积分时间,测时
精度能达到约100纳秒的量级。这样的精度是低于原子钟的,主要是
因为脉冲星的测时噪声,以及我们在得到脉冲到达时间过程中引入
的误差,比如太阳系历书的误差以及星际介质的影响。然而由于
时间标准的偏差将在所有脉冲星的测时残差中导致相同的信号,
于是我们可以通过研究多颗脉冲星的测时残差中的相关性来寻找
时间标准的偏差。Hobbs et al. (2012)\supercite{hcm+12}发展了
确定多颗脉冲星的测时残差中的相关信号的方法。使用PPTA的数据,
他们还原了国际原子钟时间的自1994年以来的变化。未来IPTA的
数据有望进一步提高脉冲星时间标准的精度。

\subsection{脉冲星导航}

太阳系内的精确星际导航对于现有和未来的航天项目都极为重要。目前
最常用的星际导航手段是使用地面的大型射电望远镜网络跟踪航天器。
这样的导航不是航天器自主的,于是随着航天器到地球的距离的增加,
信号在望远镜和航天器间的延迟越来越大,导航的精度越来越低。

作为太阳系外的天体,脉冲星的稳定的脉冲星信号可以被用来进行星际导航。
考虑我们在航天器上对脉冲星的脉冲信号进行接受并且测量脉冲到达时间,为了与
脉冲星的自转减慢模型和双星轨道参数模型进行比较,我们需要将脉冲到达时间
转换到太阳系质心坐标系,而这一转换以来于航天器的位置。如果航天器的实际
位置与我们的预期有偏离,那么我们将在脉冲星的测时残差中观察到额外的
噪声。于是通过在航天器上监测已知测时行为的脉冲星,并且通过修正航天器
的位置来使测时噪声最小,我们可以实现时时给出航天器的位置。

Wallace (1998)\supercite{wal88}的计算显示,对于大部分航天器,要使用射电
脉冲星来进行导航需要的射电望远镜将过大。而在光学波段能观测到的脉冲星
太少,因此使用光学观测进行导航也是不现实的\supercite{sg01}。最有希望的
是观测波段是X射线\supercite{cb81},主要是因为所需要的X射线望远镜的尺寸
将远小于射电和光学望远镜\supercite{she05}。

Deng et al. (2013)\supercite{dhy+13}使用PPTA项目的毫秒脉冲星研究了
使用毫秒脉冲星进行星际导航的新方法。他们的结果显示,通过地面射电望远镜
确定的脉冲星精确的自转和轨道参数对于脉冲星导航极为重要。同时脉冲星
导航的精度还依赖与监测的脉冲星数目、观测的数目、脉冲星测时的精度以及
时时计算需要的时间。使用四颗实际的毫秒脉冲星,他们得到从地球到火星
的航天器的导航精度可以达到约20公里。因此,脉冲星导航提供了一个切实可行
的、航天器自主的导航方式。随着未来望远镜、时钟稳定性和计算能力的提高,
脉冲星导航有望提供亚公里量级精度的星际导航。

\section{中国脉冲星研究的未来}

目前中国的脉冲星研究在多波段观测方面与国际领先水平还有差距,
这主要是由于缺乏一流的观测设备导致的。然而这一状况有望在
不远的将来得到改变。
%
我国已经启动两个与脉冲星研究紧密相关的大科学工程, 即“500米口径
球面射电望远镜”(Five-hundred-meter Aperture Spherical Telescope,
FAST)和“空间硬X射线调制望远镜”(Hard X-ray Modulation Telescope,HXMT)。
FAST利用我国贵州省黔南州平塘县大窝凼洼地的独特地形条件而建,
直径五百米,有效接受口径将达到三百米。2016年建成后, FAST将
成为全球上口径最大的单口径射电望远镜。与天线阵相比,FAST
不仅提供了大接受面积,还极大地简化了数据处理和校准的难度,
因此是射电脉冲星研究最理想的设备,有望发现上千颗射电辐射非
常微弱的脉冲星并极大地提高脉冲星测时的精度。这些发现和进步
将革新我们对于银河系内脉冲星的族群、致密物质状态、星际磁场
和介质等方面的认识,并有可能帮助我们探测到引力波辐射。
HXMT预计于2015年发射,它将是我国第一颗天文科学卫星。它包括高
能探测器(有效探测面积约5100\,cm$^2$,覆盖能区:20-250\,keV),
中能探测器(952\,cm$^2$,5-30\,keV)和低能探测器(384\,cm,1-15\,keV),
这些探测器对孤立或吸积脉冲星的非热X射线辐射敏感,有望帮助
我们区分不同的模型、理解极端现象的能量来源以及脉冲星的本质。

要发挥这些望远镜的能力,我们还有许多挑战需要克服。以射电脉冲星
的研究为例,对于FAST这样的单口径大型望远镜,射电干扰是非常
严峻的问题,不管对于脉冲星的搜寻还是高精度的测时。一方面
我们需要对多种射电干扰对于观测的影响有比较清楚的认识,另一方
新的硬件和软件层面的消除射电干扰的方法研究和发展。除了射电
干扰,望远镜的校准也非常重要,尤其对于高精度的脉冲星测时。
脉冲星的射电辐射是高度偏振的,如果不能很高地校准望远镜,有可能
导致脉冲轮廓随时间的变化从而影响测时精度和脉冲星测时阵列科学
目标的实现。除此之外,望远镜的时间系统、数据的储存以及处理
也是未来大型望远镜需要解决的问题。

除了我国的FAST和HXMT外,国际上还有多个已经运行和正在筹备的与
脉冲星研究紧密相关的大型设备,这些设备也为我们提高了很好的
机遇:1) 平方公里阵(Square Kilometre Array,SKA)
是多国合作的、下一代的射电干涉望远镜,中国已经正式加入这一国
际合作。SKA设计为由3000个口径15米的射电望远镜组成的五臂阵列干涉
仪,从阵列核心到边缘的距离达3000公里,建成后将成为世界上最大的、
最灵敏的射电望远镜;2) 核谱望远镜阵(Nuclear Spectroscopic Telescope Array,NuSTAR)
能在硬X射线(频段6-79\,keV)波段提高空间和谱分辨率,并已于2012年
6月13日发射升空;3) 中子星内部组分探险者(Neutron Star Interior Composition 
Explorer,NICER)为美国NASA计划建设的大面积聚焦的X射线计时
望远镜,预计在2016年放置于国际空间站,能够得到0.2-12\,keV能区的
转动相位分离谱并进行X射线导航试验;4) 除了电磁波段外,包括LIGO
和Virgo在内的千赫兹引力波望远镜也有望给出单颗脉冲星持续引力波辐
射信号,甚至探测到超新星爆发和脉冲星双星合并时产生的引力波爆发事件。
这些引力波信息将进一步限制脉冲星物态。

\pkuthssffaq

