%%%%%%%%%%%%%%%%%%%%%%%%%%%%%%%%%%%%%%%%%%%%%%%%%%%%%%%%%%%%%%%%%%%%%%%%%
%
%   LaTeX File for Doctor (Master) Thesis of Peking University
%   LaTeX + CJK     北京大学博士(硕士)论文模板
%   Based on Wang Lei's Template for THU
%   Version: 1.00
%   Last Update: 2005-05-25
%
%%%%%%%%%%%%%%%%%%%%%%%%%%%%%%%%%%%%%%%%%%%%%%%%%%%%%%%%%%%%%%%%%%%%%%%%%
%   Copyright 2004-2005  by  Ying Pan       (yeying_pan@yahoo.com.cn)
	%%%%%%%%%%%%%%%%%%%%%%%%%%%%%%%%%%%%%%%%%%%%%%%%%%%%%%%%%%%%%%%%%%%%%%%%%

	\chapter{固态夸克星模型与液固相变潜热}
	\label{chap2}

	冷夸克物质的费米能远高于热能,主要的自由度是夸克自由度,它们的物态至今仍不能很好理解,这一方面是由于在低能下夸克之间的强相互作用的
	非微扰效应,另一方是由于多体问题的困难。中子星作为最致密的一种天体,其中心的密度高达几倍核物质密度,这无疑为我们提供了一个讨论夸克
	物质物态的绝佳场所。有相对论重离子碰撞实验的证据显示,在热夸克胶子等离子体中,夸克间的相互作用很强\ucite{Shuryak09},那么自然地当温
	度降低时,夸克之间的相互作用可能更强。相对论性的夸克物质基态可能不是费米气\ucite{Xu09},于是夸克可能集团,而如果夸克团块之间的剩余
	强相互作用势能高于它们的动能,则夸克团块将被束缚在势垒中,形成固态夸克物质,我们推测天体物理中的冷夸克物质可能处于这样的固态\ucite{Xu03}。

	由于目前我们尚不能得到冷夸克物质的相对论性状态方程,通常只能使用一些唯象模型来描述,来小禹在~2009~年的工作中使用了~Lennard-Jones~势来
	讨论固态夸克星\ucite{Lai09},得到了夸克星的状态方程。基于来小禹的固态夸克星模型,我们可以尝试估算夸克星从液态相变到固态时放出的相变
	潜热以及相变温度,并将估算结果应用于伽玛暴余辉平台的解释。


	\section{固态夸克星模型}
	\label{chap2:sec1}

	假设夸克星内部的单个夸克团块是不带色的,于是我们可以将夸克星内部的夸克团块与惰性气体中电中性的的惰性气体分子类比。
	我们知道惰性气体分子之间的相互作用能很好地用~Lennard-Jones~势来刻画:
	\begin{equation}
	u(r)=4U_{0}[(\frac{r_{0}}{r})^{12}-(\frac{r_{0}}{r})^{6}]
	\end{equation}
	其中,$U_{0}$ 是势井深度,$r_{0}$ 代表相互作用的范围。我们假设这样一个短程排斥,长程吸引的相互作用势也能描述夸克团块之间的相互作用,那么
	当夸克团块间的相互作用势井足够深,以至于能将夸克团块束缚于势垒内时,夸克物质就将固化,形成固态夸克物质。由于色相互作用远远强于电磁相互
	作用,$U_{0}$ 和 $r_{0}$ 两个参数在冷夸克物质中将会不同于惰性气体中,并且决定了固态夸克物质的状态。

	用~Lennard-Jones~势来刻画的固态夸克星模型完全不同于用~MIT bag model~等模型来刻画的夸克物质模型。相比于在传统模型中夸克物质的基态是费米气,
	在~Lennard-Jones~势所描述的固态夸克星模型中,夸克团块是非相对论性的粒子,并且预言了一个更硬的状态方程,意味着夸克星的最大质量将更大,
	($>2M_{\odot}$)。图~\ref{fig:chap01:mass}~和图~\ref{fig:chap01:radius}~展示了在 $U_{0}=50$ MeV 和 $U_{0}=100$ MeV 时固态夸克星的状态方程和质量半径关系\ucite{Lai09}。

	\begin{figure*}
	\includegraphics[width=3 in]{f1.eps}
	\caption{状态方程,$N_{\rm q}=3$,$U_0=50$ MeV (蓝色实线),$U_0=100$ MeV
		(蓝色虚线); $N_{\rm q}=18$,$U_0=50$ MeV (红色虚点线),$U_0=100$ MeV
			(红色点线)。夸克质量为 $m_{\rm s}=100$ MeV , 强相互总用耦合常数为
			$\alpha_{\rm s}=0.3$,口袋常数为 $B=60 \rm MeV / fm^{-3}$ 的口袋模型
			的结果(细线),所用的表面密度为 $\rho_{\rm s}=2\rho_0$,$\rho_0$ 是
			核饱和密度。}
			\label{fig:chap01:mass}
			\end{figure*}

			\begin{figure*}
			  \includegraphics[width=3 in]{f2.eps}
				\caption{质量半径关系和质量中心密度关系。$N_{\rm q}=3$,$U_0=50$ MeV (蓝色实线),
					$U_0=100$ MeV (蓝色虚线)。$N_{\rm q}=18$,$U_0=50$ MeV (红色虚点线),
						$U_0=100$ MeV 红色点线,表面密度均取 $\rho_{\rm s}=2\rho_0$。}
						\label{fig:chap01:radius}
						\end{figure*}


						\section{液固相变潜热}
						\label{chap2:sec2}

						基于用~Lennard-Jones~势来描述的固态夸克星模型,我们可以尝试估算夸克星由液态相变到固态的过程中放出的相变潜热及相变温度。这样的相变过程有可能发生
						在夸克星形成的初期,当夸克星由高温逐渐冷却时,例如本文中讨论的伽玛暴余辉阶段。在我们所考虑的固态夸克星模型下,夸克团块作为非相对论性粒子,
						由之间的类似于~Lennard-Jones~势的剩余色相互作用束缚在一起,与惰性气体非常相似,与常见的物质也有很多相似之处。鉴于我们很难从分子动力学的方法
						直接出发,模拟夸克物质从液态相变到固态的过程,并得到相变潜热和相变温度,我们考虑通过与惰性气体,以及常规物质的类比,尤其是参考各种物质的相变潜热和
						相互作用强度的数据,从量级上估算夸克星由液态相变到固态时释放的相变潜热,这对于我们讨论伽玛暴余辉的能量注入是十分有意义的。

						在下表中,我们列出了一些常见物质和惰性气体的熔解热,气化热,势能以及势能与熔解热之比的实验数值,其中熔解热正对应于从液态相变到固态的相变潜热,
						而势能的数值对于有升华热数据的物质就是升华热,对于没有升华热数据的物质则是熔解热和气化热之和。我们看到,对于大部分物质,包括惰性气体,势能与熔
						解热之比在几十到一百之间,特例是氦的比值较高,这可能与氦复杂的物态有关。考虑到夸克团块间的相互作用与惰性气体相似,且夸克团块的质量不大,我们取
						势能与熔解热之比为 $20\sim100$ 进行估算。而在上一节讨论的固态夸克星模型下,参考来小禹的工作,取势井深度 $U_{0}=100$ MeV,那么我们可以
						估算得到单个夸克团块由液态相变到固态是放出的相变潜热为 $1\sim5$MeV。假设夸克星的质量是一个太阳质量,
						$M_{\odot}\approx2\times10^{33}$g,约包含重子数目为 $10^{57}$ 个,这样我们得到一个太阳质量的夸克星由液态相变到固态放出的相变潜热约为 $E\approx10^{51}$ ergs,这个估算值基本上就是伽玛暴余辉释放能量的典型值,也就是说夸克星由液态相变到固态时释放的相变潜热完全足够提供伽玛暴余辉平台需要的能量注入。

						\begin{table}[htb!]
						\caption{一些常见物质和惰性气体的熔解热,气化热,势能以及势能与熔解热之比}
						\centerline{\begin{tabular}{|c|c|c|c|c|c|}\hline
							 &  熔解热 kcal/mol  & 气化热 kcal/mol   & 势能 kcal/mol  &   势能/熔解热  \\
								 \hline
								 $He$ & 0.0033 & 0.0194 & 2.2944   & 695.27 \\
								 \hline
								 $Ne$ & 0.0801 & 0.422  & 9.1776   & 114.58 \\
								 \hline
								 $Xe$ & 0.5495 & 3.02   & 12.6192  & 22.96  \\
								 \hline
								 $Rn$ & 0.69   & 4.01   & 19.5024  & 28.26  \\
								 \hline
								    &        &        &          &        \\
										\hline
										$Al$ & 2.56   & 69.5   & 78       & 30.47  \\
										\hline
										$Cs$ & 0.499  & 16.198 & 18.3     & 36.67  \\
										\hline
										$Cu$ & 3.17   & 72.74  & 81       & 25.55  \\
										\hline
										$Fe$ & 3.63   & 83.68  & 99.5     & 27.41  \\
										\hline
										$Hg$ & 0.5486 & 14.13  & 14.65    & 26.7   \\
										\hline
										$Na$ & 0.622  & 23.285 & 25.75    & 41.4   \\
										\hline
										$Si$ & 12     & 85.8   & 107.7    & 8.98   \\
										\hline
										$C$  & 25     &        & 171.29   & 6.85   \\
										\hline
										$CO$ & 0.2    & 1.444  & 1.644    & 8.22   \\
										\hline
										$CO_{2}$  & 1.99      &        & 6.03   & 3.03  \\
										\hline
										$H_{2}O$  & 1.436      &  9.717    & 11.153   & 7.77  \\
										\hline
										$H_{2}O_{2}$  & 2.987   &  10.53   & 12.34   & 4.13   \\
										\hline
										$CaCl_{2}$  & 6.8      & 56.2       & 77.5   & 11.4  \\
										\hline
										\end{tabular}}
										\end{table}

										另一方面,根据~Lindemann~经验定则,当固体中原子振动强度的方均根与平衡的相邻原子距离的比值大于一定的临界值时固体熔解\ucite{Lindemann10},我们可以
										估算夸克星相变的温度。~P. Mohazzabi~和~F. Behroozi~在~1987~年的工作中得到,对于惰性气体,原子振动强度的方均根与平衡的相邻原子距离的比值有表达式:
										\begin{equation}
										\frac{\langle u^{2} \rangle}{R_{0}^{2}}=\frac{\varepsilon_{0}/kT}{4(\exp(-\alpha/kT)-\exp(-\varepsilon_{0}/kT))}\int_{\alpha/\varepsilon_{0}}^{1}[(1-x^{1/2})^{-1/6}-(1+x^{1/2})^{-1/6}]^{2}\exp(-\varepsilon_{0}x/kT)/rm{d}x
										\end{equation}
										其中 $x=1+(\varepsilon/\varepsilon_{0})$,$\varepsilon_{0}=U_{0}S_{6}^{2}/6S_{12}$,$k$ 是玻尔兹曼常数,对于简单的面心立方晶体,$S_{6}=14.45392$,
										$S_{12}=12.13188$\ucite{Moha87},带入惰性气体的参数后发现比值很好地与经验定则符合。对于我们的估算,可以直接参考惰性气体以及普通物质的势能与
										热能的比值,$\Gamma=U_{0}/kT$,从而估算夸克星相变的温度。下表列出了惰性气体的相关参数,可以发现,势能与热能的比值大约在 $1.75$ 左右。而对于其他
										物质,例如对于单一组分的等离子体,液体固化时库伦势与热能的比值是 $\Gamma\approx175$\ucite{DeWitt01},对于混合组分的等离子体,$\Gamma\approx233$
										\ucite{Horowitz07}。因此,若取势井深度 $U_{0}=100$MeV,我们可以直接估算得到夸克星的相变温度大约在 $1\sim10$MeV。

										\begin{table}[htb!]
										\caption{惰性气体的相变温度和势能}
										\centerline{\begin{tabular}{|c|c|c|c|}\hline
											 &  熔点 K  & $U_{0}/\rm{k}$ K   & $U_{0}/\rm{k}T$  \\
												 \hline
												 $Ne$ & 24.5 & 45.86  & 1.87 \\
												 \hline
												 $Ar$ & 83.8 & 142.095   & 1.70  \\
												 \hline
												 $Kr$ & 115.8   & 201.9   & 1.74  \\
												 \hline
												 $Xe$ & 161.4   & 281.0   & 1.74  \\
												 \hline
												 \end{tabular}}
												 \end{table}

												 处在液态到固态相变过程中的夸克星将保持稳定的相变温度,同时向外辐射相变潜热,我们假设夸克星的热辐射是一个灰体辐射,辐射效率是 $\eta\ll1$,则我们可以
												 估算夸克星辐射相变潜热的时标:
												 \begin{equation}
												 t=\frac{E}{\sigma T^{4} 4\pi R^{2}\eta}
												 \end{equation}
												 其中,$E=10^{51}$ ergs,$\sigma$ 是~Stefan-Boltzman~常数,$R=10$km 是夸克星的半径,考虑到 $\eta\ll1$,带入 $T\approx1$MeV 计算可得,相变潜热的辐射
												 时标 $t\approx1000$ s,这是与伽玛暴余辉的平台持续时间基本一致的。


												 \section{总结}
												 \label{chap2:sec6}

												 我们在以~Lennard-Jones~势来描述的固态夸克星模型下估算了夸克星由液态相变到固态的相变温度以及相变过程中所放出的相变潜热,并估算了相变能量以灰体辐射的
												 形式辐射的时标。我们得到的相变潜热总能量,$E=10^{51}$ ergs,完全足够提供伽玛射线暴余辉的平台阶段需要的能力注入,而 $t\approx1000$ s 的能量辐射时标也与平台持续的时间基本一直。

												 夸克星作为致密中子星的一种可能本质,很可能在由大质量恒星塌缩或双星合并导致的伽玛射线暴中形成,而在夸克星从诞生后逐渐冷却的过程中,由液态向固态的相变
												 过程的出现也是极有可能的。这样的相变过程可能发生在爆发之后的余辉阶段,作为一种中心机制为余辉提供能量注入,我们的以上估算已经说明了这种中心机制的可行性。
												 这样的中心机制能在很多方面解释一些我们在原来的框架下尚无法理解的余辉光变曲线的特征,一方面由于相变过程中夸克星的温度是不变的,因此能量会持续而稳定地注
												 入余辉,很自然地产生余辉的平台,另一方面,一旦相变过程结束,相变潜热的注入将立刻停止,从而产生平台期之后的快速,陡峭的下降阶段,而这样的光变特征目前尚
												 无法被其他的中心机制所解释。

												 %%%%%%%%%%%%%%%%%%%%%%%%%%%%%%%%%%%%%%%%%%%%%%%%%%%%%%%%%%%%%%%%%%%%%%%%%
												 %
												 %   LaTeX File for Doctor (Master) Thesis of Peking University
												 %   LaTeX + CJK     北京大学博士(硕士)论文模板
												 %   Based on Wang Lei's Template for THU
												 %   Version: 1.00
												 %   Last Update: 2005-05-25
												 %
												 %%%%%%%%%%%%%%%%%%%%%%%%%%%%%%%%%%%%%%%%%%%%%%%%%%%%%%%%%%%%%%%%%%%%%%%%%
												 %   Copyright 2004-2005  by  Ying Pan       (yeying_pan@yahoo.com.cn)
	%%%%%%%%%%%%%%%%%%%%%%%%%%%%%%%%%%%%%%%%%%%%%%%%%%%%%%%%%%%%%%%%%%%%%%%%%

	\chapter{火球模型与夸克星相变潜热注入余辉} \label{chap3}

	在估算了夸克星由液态相变到固态时放出的相变潜热,以及能量注入的时标之后,我们有必要定性地讨论一下相变潜热以怎样的方式
	注入伽玛暴余辉的问题。在这一章中,我们将首先简要地回顾伽玛暴的标准火球模型,主要参考了~Piran~在1999年的综述\ucite{Piran99},
	然后在火球模型的框架下,我们将定性地讨论夸克星相变潜热注入伽玛暴余辉的过程。

	\section{火球模型}

	本节对于伽玛暴标准火球模型的回顾将主要关注原始伽玛暴外流的加速过程和辐射机制,包括经典火球的相对论性运动,火球的演化过程和
	伽玛暴非热辐射的产生三个部分。

	\subsection{经典火球的相对论性运动}

	伽玛暴火球的相对论性运动问题的讨论最初源于对伽玛暴的致密性问题的思考\ucite{Piran99},简单地说,就是在伽玛暴爆发时,巨大的能量
	聚集在极小的范围之内,辐射区的光深将极大,于是理论上应该观测到一个黑体辐射,而事实上却观测到了非热辐射。考虑一个观测流量为,
	$F$,的典型爆发,来自一个距离为 $D$ 的各向同性辐射的源,那么释放的总能量约为:
	\begin{equation}
	E=4\pi D^{2}F=10^{50}\rm{ergs}(\frac{D}{3000\rm{Mpc}})^{2}(\frac{F}{10^{-7}\rm{ergs/cm^{2}}})
	\end{equation}
	伽玛暴的典型光变时标 $\delta T\approx10$msec 意味着源的尺度 $R_{i}<c\delta T\approx3000$km。假设有比例为 $f_{p}$ 的光
	子能发生 $\gamma\gamma\rightarrow e^{+}e^{-}$ 的反应,那么这个过程的光深可以用下式估算:
	\begin{equation}
	\tau_{\gamma\gamma}=\frac{f_{p}\sigma_{T}FD^{2}}{R^{2}_{i}m_{e}c^{2}}=10^{13}f_{p}(\frac{D}{3000\rm{Mpc}})^{2}(\frac{F}{10^{-7}\rm{ergs/cm^{2}}})(\frac{\delta T}{10\rm{msec}})^{-2}
	\end{equation}
	其中,$\sigma_{T}$ 汤姆森散射截面。可以看到光深非常大,这样的辐射区辐射应该是热谱,与观测到的非热谱相矛盾。

	克服致密性问题的一个最合理的途径就是假设原始爆发产生的火球是以极端相对论的速度向观测者运动的。考虑辐射源以相对论性
	速度向一个处于静止参考系的观测者运动,$\gamma=1/\sqrt{1-\frac{\upsilon^{2}}{c^{2}}}\gg1$。那么观测到的能量为 $h\nu_{obs}$
	的光子已经发生了蓝移,其在源的参考系中发射时的能量为 $\approx h\nu_{obs}/\gamma$。这样在源的参考系中的光子能量比观测到的光
	子能量低,从而能发生 $\gamma\gamma\rightarrow e^{+}e^{-}$ 反应的光子将更少,因此在源的参考系中 $f_{p}$ 将减小一个
	因子 $\gamma^{-2\alpha}$,其中 $\alpha$ 是高能光谱的谱指数。同时,相对论效应使辐射发出的半径,$R_{e}<c\delta T\gamma^{2}$,
	比原来的估计,$R_{e}<c\delta T$,增大了一个因子 $\gamma^{2}$,从而我们得到:
	\begin{equation}
	\tau_{\gamma\gamma}\approx\frac{10^{13}}{\gamma^{4+2\alpha}}f_{p}(\frac{D}{3000\rm{Mpc}})^{2}(\frac{F}{10^{-7}\rm{ergs/cm^{2}}})(\frac{\delta T}{10\rm{msec}})^{-2}
	\end{equation}
	如果要克服致密性问题,我们需要 $\gamma>10^{13/(4+2\alpha)}\approx100$。这样极端的相对论性运动是我们在任何天体物理过程
	中都没有观测到过的,而这样的极端相对论性运动后来被~VLBA~的观测证实了。

	\subsection{火球的演化过程}

	在讨论了火球的相对论性运动之后,以下我们简要地回顾火球的演化过程。我们考虑,伽玛暴的原始火球是纯辐射的,如果初始温度
	足够高,那么正负电子对将产生,并且由于光深很大,光子不能自由地逃出火球。电子对与光子通过双光子过程耦合在一起,形成类似
	理想流体的等离子体,满足状态方程 $p=\rho/3$。理想流体在自身内部的压力下膨胀,随着膨胀理想流体将逐渐冷却,$T\propto R^{-1}$,
	其中 $T$ 是流体的温度,$R$ 是半径。当温度低于电子对产生的临界温度时,电子对将开始湮灭,当温度降到 $20$ keV 时电子对的数
	目将变得很稀少,等离子体变得透明,光子可以自由地逃逸。同时火球仍然在加速膨胀,能量守恒要求火球向外运动的洛仑兹因子满足
	$\gamma\propto R$。逃逸的光子在火球参考系中的温度约为 $T=20$ keV,然而被观测到时将产生蓝移,$T_{obs}\propto\gamma T$,
	由于 $T\propto R^{-1}$,$\gamma\propto R$,我们可以观测到的光子温度正比于初始温度 $T_{0}$。

	除了正负电子对和光子以外,火球还包含一些重子物质,这可能是在爆发阶段带入的,也可能是存在于源附近的空间中的重子物质当火
	球膨胀时被带入的。重子物质对火球的演化将产生两方面的影响,一方面是增加火球的不透明度,延迟光子的逃逸,另一方面是重子将
	和火球一起加速,转移一部分的辐射能到重子的动能,后一方面的影响将在火球的演化过程中起至关重要的作用,导致伽玛暴产生的内
	激波正是发生在火球的物质主导阶段。考虑了重子的影响,火球的膨胀将被分为两个阶段,辐射主导的阶段和物质主导的阶段,两者的
	过度发生在:
	\begin{equation}
	R_{\eta}=\frac{R_{i}E}{Mc^{2}}
	\end{equation}
	此时火球平均的洛仑兹因子为,$\gamma\approx E/Mc^{2}$,其中 $M$ 是重子物质的质量,此时所有的能量几乎都转化为重子物质
	的动能。物质主导阶段又分为两个阶段,一开始火球内部的壳层在自身的参考系中以固定的宽度向外膨胀,宽度 $\sim(E/Mc^{2})R_{i}$。
	之后,当火球膨胀到 $R_{s}=R_{i}\gamma^{2}$ 时,壳层的宽度开始正比于半径增大。以上是火球演化的基本图像,以下我们将从基本的
	守恒律出发,分别考察辐射主导阶段和物质主导阶段。

	考虑一个各向同性的球对称火球,在光厚的条件下,光子和相对论性轻子(能量密度为 $e$)以及重子物质(重子密度为 $\rho$)在
	不同半径下均类似理想的统一流体,以相同的速度运动。压强,$p$,和能量密度,$e$,满足关系 $p=e/3$。这样的理想流体的能动量
	张量为:
	\begin{equation}
	T_{\mu\nu}=pg_{\mu\nu}+(p+e+\rho)U_{\mu}U_{\nu}
	\end{equation}
	粒子流四矢量:
	\begin{equation}
	N^{\alpha}=nU^{\alpha}
	\end{equation}
	其中,$n$ 是粒子数密度,对于光子气 $n=e^{3/4}$。根据相对论性的粒子数守恒和能动量张量守恒,
	\begin{equation}
	\partial^{\mu}T_{\mu\nu}=0
	\end{equation}
	\begin{equation}
	\partial_{\mu}N^{\mu}=0
	\end{equation}
	我们可以得到,
	\begin{equation}
	\frac{\partial}{\partial t}(\rho \gamma)+\frac{1}{r^{2}}\frac{\partial}{\partial r}(r^{2}\rho u)=0
	\end{equation}
	\begin{equation}
	\frac{\partial}{\partial t}(e^{3/4}\gamma)+\frac{1}{r^{2}}\frac{\partial}{\partial r}(r^{2}e^{3/4}u)=0
	\end{equation}
	\begin{equation}
	\frac{\partial}{\partial t}[(\rho+\frac{4}{3}e)\gamma u]+\frac{1}{r^{2}}\frac{\partial}{\partial r}[r^{2}(\rho+\frac{4}{3}e)u^{2}]=-\frac{1}{3}\frac{\partial e}{\partial r}
	\end{equation}
	其中,$u=\sqrt{\gamma^{2}-1}$,并且取单位 $c=1$,一个重子的质量 $m=1$。做坐标变换,
	$s=t-r$,上式可变换为:
	\begin{equation}
	\frac{1}{r^{2}}\frac{\partial}{\partial r}(r^{2}\rho u)=-\frac{\partial}{\partial s}(\frac{\rho}{\gamma+u})
	\end{equation}
	\begin{equation}
	\frac{1}{r^{2}}\frac{\partial}{\partial r}(r^{2}e^{3/4}u)=-\frac{\partial}{\partial s}(\frac{e^{3/4}}{\gamma+u})
	\end{equation}
	\begin{equation}
	\frac{1}{r^{2}}\frac{\partial}{\partial r}(r^{2}(\rho+\frac{4}{3}e)u^{2})=-\frac{\partial}{\partial s}[(\rho+\frac{4}{3}e)\frac{u}{\gamma+u}]+\frac{1}{3}(\frac{\partial e}{\partial s}-\frac{\partial e}{\partial r})
	\end{equation}
	在短暂的加速之后,可以认为火球的速度能达到极端相对论,$\gamma\gg1$,于是可以将 $\gamma^{-1}$ 当作小量,且 $\gamma\approx u$。
	在这样的条件下,以上三个方程可以得到化简,并且最后得到:
	\begin{equation}
	r^{2}\rho\gamma=\rm{const.}, r^{2}e^{3/4}\gamma=\rm{const.}, r^{2}(\rho+\frac{4}{3}e)\gamma^{2}=\rm{const.}.
	\end{equation}
	结合以上三个守恒关系,可以得到适用于辐射主导阶段和物质主导阶段,以及转换阶段的标量解。设 $t_{0}$ 和 $r_{0}$
	分别是火球加速到极端相对论速度时的时间和半径,同理对于其他量。定义 $D$:
	\begin{equation}
	\frac{1}{D}=\frac{\gamma_{0}}{\gamma}+\frac{3\gamma_{0}\rho_{0}}{4e_{0}\gamma}-\frac{3\rho_{0}}{4e_{0}}
	\end{equation}
	可以得到:
	\begin{equation}
	r=r_{0}\frac{\gamma^{1/2}D^{3/2}}{\gamma^{1/2}}, \rho=\frac{\rho_{0}}{D^{3}}, e=\frac{e_{0}}{D^{4}}.
	\end{equation}
	这些参数关系刻画了不同时间对于不同的壳层,$r$,$\rho$,$e$ 与 $\gamma$ 满足的关系,对于辐射主导和物质主导
	阶段都适用,下面我们将分别讨论两个阶段。

	(1)辐射主导阶段:

	最初,火球是辐射主导的,即 $\gamma\ll(e_{0}/\rho_{0})\gamma_{0}$,于是  $\frac{1}{D}=\frac{\gamma_{0}}{\gamma}+\frac{3\gamma_{0}\rho_{0}}{4e_{0}\gamma}-\frac{3\rho_{0}}{4e_{0}}$
	中的第一项将起主导作用,可以得到:
	\begin{equation}
	\gamma\propto r, \rho\propto r^{-3}, e\propto r^{-4}.
	\end{equation}
	这样的标量关系与纯辐射理想流体在随动参考系中的膨胀演化关系是一致的。火球壳层的宽度满足,$\Delta r\sim r/\gamma\sim \rm{constant}$,
	即在观测者看来,壳层保持一个固定的宽度向外加速运动,而这样的状态只要火球保持极端相对论性的膨胀就将持续。

	(2)物质主导阶段:

	辐射主导的阶段将一直延续到,$r\sim(e_{0}/\rho_{0})r_{0}$。当半径更大时,
	$\frac{1}{D}=\frac{\gamma_{0}}{\gamma}+\frac{3\gamma_{0}\rho_{0}}{4e_{0}\gamma}-\frac{3\rho_{0}}{4e_{0}}$
	中的第一项和最后一项将是可以比较的,$\gamma$ 将达到一个临界值 $\gamma_{f}=(4e_{0}/3\rho_{0}+1)\gamma_{0}$,
	这时就进入了物质主导阶段。在这个阶段,
	\begin{equation}
	\gamma\rightarrow constant, \rho\propto r^{-2}, e\propto r^{-8/3}.
	\end{equation}
	辐射成分已经不再发挥重要作用,光子和轻子压强不再加速火球膨胀,$\gamma$ 保持一个常数,理想流体以一个临界
	速度运动。在比较晚的阶段,脉冲冻结巡航阶段也将被破坏,因为辐射的能量密度远小于重子物质密度,$\gamma$ 保持
	不变,从而守恒方程中的 $\partial e/\partial r$ 将被忽略,并且视 $u$ 为常量,可以发现火球壳层的宽度 $\Delta r$
	将正比于时间膨胀,火球最后进入膨胀巡航阶段。值得提到的是,人们认为由于中心源释放能量的不稳定,火球的不同
	壳层的速度是不同的,于是在物质主导阶段,在脉冲冻结巡航阶段和膨胀巡航阶段,不同的壳层将可能发生碰撞,产生内激波。

	\subsection{伽玛暴非热辐射的产生}

	正如我们在讨论伽玛暴的致密星问题时提到的,完全源自原始火球的辐射一定是热辐射,这与观测到的非热辐射是不符,而要
	产生非热辐射,需要将火球径向的动能转化为内部的热能,从而在磁场下产生同步辐射。目前主流的观点认为,火球内部不同
	壳层之间的碰撞产生内激波,将火球径向的动能转化为内部热能,通过同步辐射产生伽玛暴爆发阶段的伽玛射线,而火球与外
	部介质的碰撞产生外激波,进而产生伽玛暴的余辉。这一小节中,我们简要地从相对论性非弹性碰撞和同步辐射两方面广义地
	讨论伽玛暴非热辐射的产生。

	(1)相对论性非弹性碰撞:

	考虑火球的一个壳层(用下标 $r$ 表示)追赶上了另一个壳层(用下标 $s$ 表示),发生非弹性碰撞并合并为一(用
	下标 $m$ 表示),根据能量和动量守恒,可以得到:
	\begin{equation}
	m_{r}\gamma_{r}+m_{s}\gamma_{s}=(m_{r}+m_{s}+\epsilon/c^{2})\gamma_{m}
	\end{equation}
	\begin{equation}
	m_{r}\sqrt{\gamma^{2}_{r}-1}+m_{s}\sqrt{\gamma^{2}_{s}-1}=(m_{r}+m_{s}+\epsilon/c^{2})\sqrt{\gamma^{2}_{m}-1}
	\end{equation}
	其中,$\epsilon$ 是碰撞中产生的内能。

	对于内激波的情形,两个壳层以不同的相对论性速度运动,$\gamma_{r}\gtrsim\gamma_{s}\gg1$,可以得到:
	\begin{equation}
	\gamma_{m}=\sqrt{\frac{m_{r}\gamma_{r}+m_{s}\gamma_{s}}{m_{r}/\gamma_{r}+m_{s}/\gamma_{s}}}
	\end{equation}
	合并后产生的内能,$E_{int}=\gamma_{m}\epsilon$,等于碰撞前后动能的差,
	\begin{equation}
	E_{int}=m_{r}c^{2}(\gamma_{r}-\gamma_{m})+m_{s}c^{2}(\gamma_{s}-\gamma_{m})
	\end{equation}
	动能转化为内能的效率为:
	\begin{equation}
	\eta=1-\frac{(m_{r}+m_{s})\gamma_{m}}{m_{r}\gamma_{r}+m_{s}\gamma_{s}}
	\end{equation}
	容易看到,如果希望动能转化为内能的效率高,那么要求两个壳层的速度差比较大,$\gamma_{r}\gg\gamma_{s}$,
	同时两壳层的质量相当,$m_{r}\approx m_{s}$。

	对于外激波的情形,$\gamma_{s}=1$,要使 $\epsilon\approx m_{r}/2$ 且 $\gamma_{m}\approx\gamma_{r/2}$,
	需要 $m_{s}\approx m_{r}/\gamma_{r}\ll m_{r}$,即要将一半的初始动能转化为内能,需要的外部介质质量为初始
	质量的 $1/\gamma_{r}$。对于外激波,壳层的不同部分与外部介质相互作用的时间可能相对较长,因此流体力学的
	过程也需要被考虑。如果一个冷的壳层进入外部介质,那么将会产生进入外部介质的激波,以及一个反向的进入壳层
	的激波,壳层相对于外部介质的速度以及不同区域的粒子数密度决定了激波的结构。

	(2)同步辐射

	伽玛暴的辐射过程普遍认为是同步辐射,这也被低能的光谱特征所证实。决定同步辐射的参数主要是磁场强度,$B$,
	以及电子能量分布(用电子的最小洛仑兹因子,$\gamma_{e,min}$,和假设的电子能量幂率分布的谱指数来刻画)。
	这些参数应该由激波的微观物理过程决定,然而这很难从第一性原理估计,因此通常人们定义两个无量纲的参数,
	$\epsilon_{B}$ 和 $\epsilon_{e}$,来代表不确定性。

	无量纲参数 $\epsilon_{B}$ 度量了磁场能量密度和总的热能 $e$ 的比,
	\begin{equation}
	\epsilon_{B}=\frac{U_{B}}{e}=\frac{B^{2}}{8\pi e}
	\end{equation}
	通常将 $\epsilon_{B}$ 作为一个自由参数,并且假设磁场是在空间中任意地产生的。
	另一个无量纲参数 $\epsilon_{e}$ 度量转化为无规则运动的能量占总能量的比例,
	\begin{equation}
	\epsilon_{e}=\frac{U_{e}}{e}
	\end{equation}

	考虑到电子的无规则动能来自激波加热,假设(沿用非相对论激波的考虑)电子分布满足一个关于洛仑兹因子的
	幂率关系:
	\begin{equation}
	N(\gamma_{e})\sim \gamma_{e}^(-p) \rm{for} \gamma_{e}>\gamma_{e,min}
	\end{equation}
	要使能量在很大的洛仑兹因子时
	发散,则要求 $p>2$。最小的洛仑兹因子 $\gamma_{e,min}$ 与 $\epsilon_{e}$
	以及总能量 $e\sim \gamma_{sh}nm_{p}c^(2)$ 有关:
	\begin{equation}
	\gamma_{e,min}=\frac{m_{p}}{m_{e}}\frac{p-2}{p-1}\epsilon_{e}\gamma_{sh}
	\end{equation}
	其中,$\gamma_{sh}$ 是跨越激波的相对洛仑兹因子。

	在幂率的电子分布的基础上,可以考虑同步辐射的谱的形式,而光谱正是区别不同辐射机制的重要依据。
	一个洛仑兹因子为 $\gamma_e \gg 1$ 的相对论性的电子在磁场 $B$ 中同步辐射的功率和特征频率为\ucite{Sari98}
	\begin{equation}
	\label{power}
	P(\gamma_e)=\frac 4 3 \sigma_T c \gamma^2 \gamma_e^2 \frac {B^2} {8\pi},
	\end{equation}
	\begin{equation}
	\label{freq}
	\nu(\gamma_e)=\gamma \gamma_e^2 \frac {q_e B} {2 \pi m_e c},
	\end{equation}
	其中, $\gamma^2$ 和 $\gamma$ 因子是从激波的参考系变化到观测者参考系时引入的。
	单个能量为 $\gamma_{e}m_{e}c^{2}$ 电子的瞬时同步辐射谱是一个以 $F_{\nu}\propto\nu^{1/3}$ 的
	形式上升到 $\nu_{syn}(\gamma_{e})$,然后指数下降的形式。如果电子的能量较高,那么它将快速地
	冷却直到洛仑兹因子降到 $\gamma_{e,c}$,这是电子在流体力学时标 $t$ 下冷却的洛仑兹因子,根据
	$\gamma\gamma_{e,c}m_ec^2=P(\gamma_{e,c})t$ 可以的到
	\begin{equation}
	\gamma_c= \frac {6 \pi m_e c} {\sigma_T \gamma B^2 t}=
	\frac {3 m_e} {16 \epsilon_B \sigma_T m_p c} \frac 1 {t \gamma^3 n},
	\end{equation}
	因此对于考虑冷却过程的电子,必须要考虑一个时间积分的谱:$F_{\nu}\propto\nu^{-1/2}$ 从 $\nu_{syn}(\gamma_{e,c})$
	到 $\nu_{syn}(\gamma_{e})$。

	要计算由所有电子产生的总光谱,需要对 $\gamma_{e}$ 积分,考虑有两种不同情况,快速冷却的情况,
	$\gamma_{e,min}>\gamma_{e,c}$,和缓慢冷却的情况,$\gamma_{e,min}<\gamma_{e,c}$。

	对于快速冷却的情况,所有的电子都先冷却到 $\gamma_{e,c}$,因此在频率 $\nu_{syn}{\gamma_{e,c}}$
	处能量大约是 $N_{e}P_{\nu,max}$,其中,$N_{e}=4\pi R^{3}n/3$,是激波扫过的电子总数。
	对于高能的电子,冷却总是很快,因此高能的谱总满足:
	\begin{equation}
	F_{\nu}=N[\gamma(\nu)]m_{e}c^{2}\gamma{\nu}\rm{d}\gamma/\rm{d}\nu\propto\nu^{-p/2}
	\end{equation}
	同样,对于低能的光子,冷却总是缓慢的,因此低能的光谱总满足 $F_{\nu}\propto\nu^{1/3}$。
	于是可以得到,对于快速冷却的情形,光谱有如下形式:
	\begin{equation}
	F_\nu=
	\begin{cases}
	( \nu / \nu_c )^{1/3}F_{\nu,max}, & \nu_c>\nu, \\
		( \nu / \nu_c )^{-1/2}F_{\nu,max}, & \nu_m>\nu>\nu_c, \\
		( \nu_m / \nu_c )^{-1/2}( \nu / \nu_m)^{-p/2}F_{\nu,max}, & \nu>\nu_m,
	\end{cases}
	\end{equation}
	其中,$\nu_{m} \equiv \nu_{syn}(\gamma_{e,min})$,$F_{\nu,max}\equiv N_eP_{\nu,max}/4\pi D^2$ 是
	在距离源 $D$ 观测到的峰值流量。

	对于缓慢冷却的情况,只有 $\gamma_{e}<\gamma_{e,c}$ 的光子冷却,而 $\gamma_{e}\sim\gamma_m$ 的光子
	在一定时间内不冷却,所以低能的谱满足 $F_{\nu}\propto\nu^{1/3}$。那些稍高能量的,占所有光子的大部分
	的光子,做同步辐射,光谱满足:
	\begin{equation}
	F_{\nu}=N[\gamma(\nu)]P[\gamma(\nu)]\rm{d}\nu\propto\nu^{-(p-1)/2}
	\end{equation}
	因此可以得到光谱:
	\begin{equation}
	\label{spectrumslow}
	F_\nu=
	\begin{cases}
	(\nu/\nu_m)^{1/3}F_{\nu,max},
	            & \nu_m>\nu, \\
								(\nu/\nu_m)^{-(p-1)/2}F_{\nu,max},
							            & \nu_c>\nu>\nu_m, \\
														\left( \nu_c/\nu_m \right)^{-(p-1)/2}
														\left( \nu/\nu_c \right)^{-p/2}F_{\nu,max},
														            & \nu>\nu_c.
																				\end{cases}
																				\end{equation}

																				除了以上的冷却因素外,还有同步辐射的自吸收等因素影响光谱的特征,这里不做更多讨论,可以参考~Piran~综述。
																				图~\ref{fig:chap2:Spectra}~分别展示了快速冷却和缓慢冷却两种情形的光谱。
																				\begin{figure*}
																				\centering
																				\includegraphics[height=8cm]{fig111.eps}
																				\caption{
																					  幂率分布的电子在相对论性激波中的同步辐射谱。(a) 快速冷却的情形,
																							  一般认为在伽玛暴余辉的早期出现($t<t_0$)。谱由 A,B,C,D 四个
																								  部分组成。自吸收在 $\nu_a$ 一下比较显著。特征频率, $\nu_m$,
																									  $\nu_c$,$\nu_a$ 随时间降低;箭头上方的量代表绝热的演化,下方方括号
																										  里的量代表完全的辐射的演化。(b) 缓慢冷却的情形,期待在晚期出现
																											  ($t>t_0$)。演化总是绝热的,四个部分为 E,F,G,H\ucite{Sari98}。
																				}
\label{fig:chap2:Spectra}
\end{figure*}

\section{夸克星相变潜热注入伽玛暴余辉的过程}

在上一节中,我们从经典火球的相对论性运动,相对论性火球的演化以及非热辐射的产生等方面简要地回顾了伽玛
暴火球模型的图像。在这一节中,我们将在第一章所提出的以夸克星相变潜热为能源,注入伽玛暴余辉的想法的基础
上,在伽玛暴火球模型的框架下,初略地讨论夸克星相变潜热注入伽玛暴余辉的过程,并得到定性的结论。

夸克星由液态到固态的相变过程应该发生在伽玛暴的爆发阶段之后,具体的时标依赖于夸克星形成之后的冷却过
程,但我们可以预期,相变发生时,伽玛暴的爆发过程已经结束,原始火球已经膨胀到了很远的距离,或者已经进
入外部介质,在夸克星表面到火球内边界或外部介质内边界之间形成了一个“真空”。由于原始火球已经将这个空间
内的大部分重子物质带出,我们假设这个空间里的重子物质密度很低,只有原始密度的 $1\% \sim 0.1\%$。
根据第一章的估算结果,夸克星相变释放的相变潜热约为 $E=10^{51}$ ergs,以灰体辐射的方式释放能量的时间约
为 $1000$ s,于是我们可以估算得到,能流约为 $L=10^{48}$ ergs/s,相比于爆发阶段的能流小了三个量
级,然而如果用夸克星半径 $R=10$ km 作为源的尺度来估算光深,我们可以得到:
\begin{equation}
\tau_{\gamma\gamma}=\frac{f_{p}\sigma_{T}FD^{2}}{R^{2}m_{e}c^{2}}=10^{13}f_{p}(\frac{D}{3000\rm{Mpc}})^{2}(\frac{F}{10^{-11}\rm{ergs/cm^{2}}})(\frac{R}{10\rm{km}})^{-2}
\end{equation}
可见光深仍然非常大。因此,由夸克星相变释放出的相变潜热能流仍将类似于原始火球,是由光子,电子对和少量
重子物质构成的类似理想流体的等离子体,并且在内部压强的作用下加速膨胀,不同的是火球不再是以极端相对性
速度朝静止观测者运动,同时由于相变过程相对稳定,也不会在火球内产生壳层。这样的一个由释放相变潜热形成的
火球将基本按照我们在上一节中讨论的火球演化的过程演化,也会有由光厚到光薄的过度,以及有辐射主导到物质主
导的过度。然而无论演化过程怎么样,我们期待最后能量将转化为重子物质的动能,于是我们可以简单地估算最后由
重子物质主导的火球的速度:
\begin{equation}
\gamma=L/Mc^{2}
\end{equation}
我们知道在爆发阶段,$\gamma\approx100$,而如前所述,我们考虑夸克星相变潜热注入阶段,能流比爆发阶段小
三个数量级,而重子物质质量为原始质量的 $1\%$,由此可得 $\gamma\approx10$。而这样的相对能量密度较小的
能流,以 $\gamma\approx10$ 的速度进入外部介质,根据~Zhang Bing~等人在2002年的工作\ucite{Zhang02},
将以类似于~Poynting~流的方式注入余辉。

由以上的论述,我们可以大致描绘夸克星相变潜热注入伽玛暴余辉的图像,在伽玛暴的爆发阶段结束以后,中心形成
的夸克星温度逐渐降低,当达到固化温度时,夸克星发生由液态向固态的相变,相变过程中温度稳定在相变温度,相
变潜热则以灰体辐射的方式稳定地向外辐射;辐射出的光子由于能量密度很高,光深很大,产生大量电子对,而光子,
电子对和少量的重子物质耦合在一起类似理想流体,按照爆发阶段火球演化相类似的方式演化;随着火球的膨胀,
最终能量将基本上转化为重子物质的动能,而由于潜热释放的能流较小,虽然重子物质相较于爆发阶段少很多,最后
加速达到的速度也不是很高,而这样的能流较小,速度不高的火球将以类似于~Poynting~流的形式注入余辉,也就是说
将能量较平和地转换给外部介质的物质,从而产生余辉阶段的平台;而在相变结束之后,相变潜热的注入停止,外部
介质不再得到能量注入,辐射的流量迅速降低,从而产生光变曲线平台后的快速而陡峭的下降。

