\documentclass[a4paper,10pt,onecolumn]{article}
\usepackage{epsf}
\usepackage{graphicx}
\usepackage{latexsym}
\usepackage{indentfirst}
\usepackage{lscape}
\usepackage{psfig}


\usepackage {amsmath}
\usepackage {amsxtra}
\usepackage {amsbsy}
\usepackage {amscd}

\usepackage {graphicx}
\usepackage {graphics}
\usepackage {amssymb}
\usepackage {amsfonts}
\usepackage {float}
\usepackage {indentfirst}

\usepackage {a4wide}

%\def\qui {1E 1547-54}
%\def\uu {4U~0142$+$61}
%\def\oo {1E~1048$-$59}
%\def\kes {1E~1841$-$045}
%\def\axj {AX~J1845$-$02}
%\def\rxs {1RXS~J1708$-$40}
%\def\eee {1E~2259$+$586}
%\def\xte {XTE~J1810$-$197}
%\def\cxo {CXOU~J1647$-$45}
%\def\smc {CXOU~J0100-72}
%\def\zerosei {SGR~1806$-$20}
%\def\zerozero {SGR~1900$+$14}
%\def\sedici {SGR~1627$-$41}
%\def\lmc {SGR~0526$-$66}
%\def\quil {1E 1547.0-5408}
%\def\uul {4U~0142$+$61}
%\def\ool {1E~1048$-$586}
%\def\kesl {1E~1841$-$045}
%\def\axjl {AX~J1844.8$-$0256}
%\def\rxsl {1RXS~J170849$-$400910}
%\def\eel {1E~2259$+$586}
%\def\xtel {XTE~J1810$-$197}
%\def\cxol {CXOU~J164710.2$-$455216}
%\def\smcl {CXOU~J010043.1-721134}


\setlength{\parindent}{25pt}
\newcommand{\be}{\begin{equation}}
\newcommand{\ee}{\end{equation}}
\newcommand{\ba}{\begin{array}{c}}
\newcommand{\ea}{\end{array}}
\newcommand{\bqa}{\begin{eqnarray}}
\newcommand{\eqa}{\end{eqnarray}}
\newcommand{\bqaa}{\begin{eqnarray*}}
\newcommand{\eqaa}{\end{eqnarray*}}
\renewcommand{\baselinestretch}{1.5}
%%%%%%%%%%%%%%%%%%%%%%%%%%%%%%%%%%%%%%%%%%%%%%%%%%%%%%%%%%%%%%%%%%%

\begin{document}

%%%%%%%%%%%%%%%%%%%%%%%%%%%%%%%%%%%%%%%%%%%%%%%%%%%%%%%%%%%%%%%%%%%


\frenchspacing

\linespread{1.2}


\title{A Multiband Study of Pulsar-like Compact Stars}
\author{Shi Dai \\
Supervisor: Renxin Xu}
\date{2012.4}
\maketitle

\begin{abstract}

Pulsar-like compact stars as the densest observable objects in the Universe, are not only important
in understanding diverse phenomena in high-energy astrophysics, but also significant in fundamental physics.
%
Besides the physics of gravity, the answer to the question that whether pulsar-like compact stars are neutron stars or quark stars
would have profound implications on the physics of condensed matter, the nature of strong interaction
as well as QCD phase transition.
%
%Multiband observations and studies have been carried out to understand diverse phenomena of
%pulsar-like compact stars. Although a good harvest of data have been accumulated and our
%knowledge of pulsar-like compact stars are richer than ever before, we are still far from reaching
%the final answer and future theoretical and observational studies are necessary.
%
We will make a multiband study of compact stars during my PhD. The main part
of my study will be focused on the radio and hard X-ray emission and timing properties of anomalous
X-ray pulsars (AXPs) and soft gamma-ray repeaters (SGRs). Both theoretical and observational research
will be carried out. For the theoretical part, the radiation mechanism of the radio and X-ray emission,
and the origin of glitches and quasi periodic oscillations will be investigated in the context of
``fallback disk+quark star'' model. For the observational part, I will try to distinguish different
models based on archived data and future observations.
%
Besides the study of AXPs and SGRs, I will continue my study on microlensing pulsar which aims to
determine the mass of isolated neutron stars. I will also participate radio observations of
pulsars with the $40\ \rm{m}$ radio telescope of Yunnan Observatory.

\end{abstract}

\section{Background}

Since discovered in 1967, pulsars have received a high degree of attention from not only
astrophysicists but also theoretical physicists. The strong gravitational and magnetic field and the
extremely high density of pulsars provide an unique laboratory to investigate fundamental
physics, such as the gravity theory, the state of condensed matter, the strong interaction
and the QCD phase transition.
%
Besides radio pulsars, several kinds of X-ray sources have been discovered and turned out to
be compact stars with strong magnetic field, including X-ray pulsars, compact central
objects (CCO), dim thermal neutron stars (DTN), anomalous X-ray pulsars (AXPs) and soft
gamma-ray repeaters (SGRs). Their similar properties to radio pulsars indicate that they are
a homogenous class of stars and we call them pulsar-like compact stars. Diverse and
energetic phenomena of pulsar-like compact stars have greatly refreshed our understanding of
the equation of state, population and evolution of pulsars and will attract increasing attention.
%

Multiband observation and study have been carried out to understand diverse phenomena of
pulsar-like compact stars. Although a good harvest of data have been accumulated and our
knowledge of pulsar-like compact stars are richer than ever before, there are still many
challenges. The most thorny challenges are the energy source and the origins of multitudinous
phenomena of AXPs and SGRs, which are closely related to the equation of state, magnetic field and
evolution of pulsar-like compact stars. The measurement of the mass of isolated neutron
stars is also difficult, but of great importance in studying the inner structure of pulsars.
The answer to these questions will greatly promote our understanding of the nature of
pulsar-like compact stars and hence help us investigate the gravity theory, the strong interaction
and the state of condensed matter.
%

In the follow sections, I will give brief reviews of AXPs/SGRs and microlensing pulsars.

\subsection{AXPs and SGRs}

Among pulsar-like compact stars, anomalous X-ray pulsars (AXPs) and soft gamma-ray repeaters
(SGRs) are two sorts of enigmatic pulsar-like objects. AXPs were first detected as soft X-ray
sources ($<10$ keV)~\cite{Fahlman,Seward,Israel}, and SGRs were discovered through the
detection of short bursts in the hard X-ray range~\cite{Mazets79a,Mazets79b}. People have
realized that AXPs and SGRs were a homogenous class of pulsar-like compact stars~\cite{Mereghetti2008},
since we have found the persistent X-ray counterparts of SGRs and also discovered bursts
in most AXPs.
%
After more than $10$ years of extensive multiband observations, diverse phenomena of
AXPs and SGRs have been presented to us. The underlying extreme astrophysical environment
of these phenomena have attracted increasing attention, both from the physical and
astronomical point of view~\cite{Mereghetti2008}.
%

Two prerequisite characteristics of AXPs and SGRs are the lack of evidence of binary companions
and the X-ray luminosity larger than the spin-down power, which distinguish AXPs and SGRs from
X-ray pulsars. AXPs and SGRs share many properties, such as the similar periods and period derivatives,
the narrow distribution of period, absence of radio emission and association with supernova remnants.
Table~\ref{tab-list1} and Table~\ref{tab-list2} list all the known AXPs and SGRs~\cite{Mereghetti2008}.
In this subsection, I will briefly review some important properties of AXPs and SGRs.

%%%%%%%%%%%%%%%%%%%%%%%%%%%%%%%%%%%%%%%%%%%%%%%%%%%%%%%%%%%%%%%%%%%%%%%%%%%%%%%%%%%%
%%%%%%%%%%%%%%%%%%%%%%%%%%%%%%%%%%%%%%%%%%%%%%%%%%%%%%%%%%%%%%%%%%%%%%%%%%%%%%%%%%%%%%

\begin{landscape}

\begin{table}
\caption{A list of all the known AXPs~\cite{Mereghetti2008}.}
 \centering
\label{tab-list1}
\begin{tabular}{lcccl}
\hline\noalign{\smallskip}
Name           &Hard X-rays$^{(a)}$& Soft X-rays$^{(a)}$          & Distance      & Location \\[3pt]
               &  ($>$10 keV)      & ($<$10 keV)                  & (kpc)         &  \\[3pt]
\hline
\textbf{Anomalous X--ray Pulsars}&&&&\\ [3pt]
 \hline
\smc           & -                 & P                            &  61           & SMC \\[3pt]
\cite{lam02}   &                   & \cite{lam02}                 &               &     \\[3pt]
 \hline
\uu            & P                 & P                            &  3.6          &     \\[3pt]
\cite{mer95}   & \cite{den Hartog} & \cite{Israel}                &\cite{dur06}   &     \\[3pt]
\hline
\oo            & D                 & P                            & 9             &     \\[3pt]
\cite{mer95}   & \cite{Leyder}     & \cite{Seward}                &\cite{dur06}   &     \\[3pt]
\hline
\qui           & -                 & P,T                          & 9             & SNR G327.24--0.13 \\[3pt]
\cite{gel07}   &                   & \cite{hal08}                 &\cite{cam07}   &     \\[3pt]
\hline
\cxo           &     -             & P,T                          & 3.9           & Massive Star Cluster \\[3pt]
\cite{mun06}   &                   &\cite{mun07,isr07}            &\cite{kot07}   &  Westerlund 1     \\[3pt]
\hline
\rxs           & P                 & P                            & 3.8           &      \\[3pt]
\cite{sug97}     & \cite{Kuiper06} &\cite{sug97}                  &\cite{dur06}   &      \\[3pt]
\hline
\xte           & -                 & P,T                          & 3.1           &      \\[3pt]
\cite{ibr04}   &                   &\cite{ibr04}                  &\cite{dur06}   &      \\[3pt]
\hline
\kes           & P                 & P                            & 8.5           & SNR Kes 73\\[3pt]
\cite{vas97}   & \cite{Kuiper04}   & \cite{vas97}                 & \cite{Tian08} &     \\[3pt]
\hline
\axj $^{(b)}$  & -                 & P,T                          & 8.5           & SNR G29.6+0.1 \\[3pt]
\cite{Torii}   &                   & \cite{Torii}                 & \cite{Torii}  &     \\[3pt]
\hline
\eee            & -                 & P                            & 7.5           & SNR CTB 109 \\[3pt]
\cite{mer95}   &                   &\cite{Fahlman}                & \cite{dur06}  &     \\[3pt]
\hline
 \hline
\end{tabular}
\end{table}

\end{landscape}

%%%%%%%%%%%%%%%%%%%%%%%%%%%%%%%%%%%%%%%%%%%%%%%%%%%%%%%%%%%%%%%%%%%%%%%%%%%%%%%%%%%%
%%%%%%%%%%%%%%%%%%%%%%%%%%%%%%%%%%%%%%%%%%%%%%%%%%%%%%%%%%%%%%%%%%%%%%%%%%%%%%%%%%%%%%

\begin{landscape}

\begin{table}
\caption{A list of all the known SGRs~\cite{Mereghetti2008}.}
 \centering
\label{tab-list2}
\begin{tabular}{lcccl}
\hline\noalign{\smallskip}
Name           &Hard X-rays$^{(a)}$& Soft X-rays$^{(a)}$          & Distance      & Location \\[3pt]
               &  ($>$10 keV)      & ($<$10 keV)                  & (kpc)         &  \\[3pt]
\hline
\textbf{Soft Gamma-ray Repeaters}&&&& \\ [3pt]
 \hline
\lmc           & -                 & P                            & 55            & LMC, SNR N49 \\[3pt]
\cite{cli80}   &                   & \cite{rot94}                 &               &     \\[3pt]
\hline
\sedici        & -                 & D,T                          & 11            & \\[3pt]
\cite{Woods99} &                   & \cite{Woods99}               & \cite{cor99}  &     \\[3pt]
\hline
\zerosei       & D                 & P                            & 15            & Massive Star Cluster \\[3pt]
\cite{lar86}   &\cite{Mere05}      &\cite{Kouveliotou}            & \cite{cor04}  &     \\[3pt]
\hline
\zerozero      & D                 & P                            & 15            & Massive Star Cluster \\[3pt]
\cite{Mazets79b} & \cite{Gotz}     &\cite{hur99e}                 & \cite{vrb00}  &     \\[3pt]
\hline
 \hline
\end{tabular}


 \textbf{Notes:}

$^{(a)}$ D = detection; P = pulsations detected; T = transient


$^{(b)}$ Candidate AXP (no $\dot P$ measurement)

\end{table}

\end{landscape}

%%%%%%%%%%%%%%%%%%%%%%%%%%%%%%%%%%%%%%%%%%%%%%%%%%%%%%%%%%%%%%%%%%%%%%%%%%%%%%%%%%%%
%%%%%%%%%%%%%%%%%%%%%%%%%%%%%%%%%%%%%%%%%%%%%%%%%%%%%%%%%%%%%%%%%%%%%%%%%%%%%%%%%%%%%%

\subsubsection{Spectral properties}

%AXPs were first detected as persistent X-ray sources, and the X-ray counterparts of
%SGRs have also been found with similar flux.
Although the distance to the AXPs and SGRs is hard to be determined accurately, roughly
estimations are available according to the X-ray absorption and are supported in some
cases by the distance estimates of the associated SNRs. Characteristic distances of at
least a few kpc indicate the typical luminosities in the range $10^{34-36}\ \rm{erg\ s^{-1}}$,
which is larger than the spin-down power inferred from their period and period derivatives~\cite{Mereghetti2008}.
%

AXPs have soft spectra below $10$ keV, which can be fitted by a combination of a steep
power-law (photon index $\sim3-4$) and a blackbody with temperature $kT\sim0.5$ keV.
The spectra of SGRs below $10$ keV are generally harder than that of AXPs, and are well
fitted with power-laws of photon index $\sim2$. People also tried to fit the spectra
of AXPs with two blackbody (Fig. $1$)~\cite{Mereghetti2008,hal05} and have found blackbody-like
components in good quality spectra of SGRs (Fig. $2$)~\cite{Mereghetti2008,Mere05b,Mere06}.
These results indicate that the soft X-ray emission is of thermal origin and the emerging
spectrum is more complex than a simple blackbody~\cite{Mereghetti2008}.
%


%%%%%%%%%%%%%%%%%%%%%%%%%%%%%%%%%%%
\begin{figure}
\centering
 \vbox{
\rotatebox{-90}{\includegraphics[width=8cm]{j1810_spec_hal05a.eps}}
\rotatebox{-90}{\includegraphics[width=8cm]{j1810_spec_hal05b.eps}}}
%\psfig{figure=j1810_spec_hal05a.eps,angle=0,width=4cm}
%\psfig{figure=j1810_spec_hal05b.eps,angle=0,width=4cm}}
\caption{X-ray spectrum of \xte\ measured with the XMM-Newton
EPIC instrument (from \cite{hal05}). Equivalently good fits are
obtained with a power law plus blackbody model (top panel) or
with the sum of two blackbodies (bottom panel).}
\end{figure}
%%%%%%%%%%%%%%%%%%%%%%%%%%%%%%%%%

%%%%%%%%%%%%%%%%%%%%%%%%%%%%%%%%%%%
\begin{figure}
\centering
\rotatebox{-90}{\includegraphics[width=8cm]{sgr1806_spec_mer05c.eps}}
%\psfig{figure=sgr1806_spec_mer05c.eps,angle=-90,width=8.5cm}
\caption{XMM-Newton EPIC spectrum of \zerosei\ fitted with a power
law plus blackbody model (from \cite{Mere05b}). The blackbody
component is the lower curve.}
\end{figure}
%%%%%%%%%%%%%%%%%%%%%%%%%%%%%%%%%

The hard X-ray emissions from AXPs and SGRs are discovered by the INTEGRAL satellite only
a few years ago~\cite{Kuiper04,den Hartog,Mere05c,Gotz}. It is surprising considering
the soft spectra of AXPs below $10$ keV, and even more interesting is that the spectra of
AXPs above $20$ keV are rather hard while the spectra of SGRs are steeper. The flat spectra
indicate that the energy released in the hard X-ray range is a significant fraction of
the total energy output from these sources. The origin of the hard X-ray tails is still
unclear, and we will discuss later.
%

\subsubsection{Variability properties}

Short bursts are the characteristic property of SGRs, which led to the discovery of this
class of high-energy sources. SGRs emit short bursts in the hard X-ray band during periods
of activity. The peak luminosity of short bursts can reach $\sim10^{42}\ \rm{erg\ s^{-1}}$,
and the duration is $\sim0.01-1\ \rm{s}$. Most of the bursts consist of single or a few
pulses which rise fast and decay slowly. Some examples of bursts light curves are shown in
Fig. $3$ (from ~\cite{Mereghetti2008}). The spectra of short bursts above $\sim15$ keV can be
well fitted by optically thin thermal bremsstrahlung models with $kT\sim30-40$ keV, but such
a model can not extend to lower energy ($\sim1-2$ keV)~\cite{Feni94}. In order to fit the
broad band spectrum from $1$ to $100$ keV, two blackbody models with temperatures $kT_{\rm{1}}\sim2-4$
keV and $kT_{\rm{2}}\sim8-12$ keV are applied (e.g. ~\cite{Feroci}). Short bursts from AXPs
have been detected by RXTE~\cite{Kaspi00,Kaspi2003}, which confirmed the link between these two
classes of objects.

%%%%%%%%%%%%%%%%%%%%%%%%%%%%%%%%%%%
\begin{figure}
\centering
\rotatebox{-90}{\includegraphics[width=8cm]{sgr1806_bursts_goe04.ps}}
%\psfig{figure=sgr1806_bursts_goe04.ps,angle=-90,width=12cm}
\caption{Short bursts from \zerosei\ observed with the IBIS
instrument on board INTEGRAL (from \cite{Gotz04}). Top panels:
light curves in the soft energy range S=15-40 keV. Middle panels:
light curves in the hard energy range H=40-100 keV. Bottom panel:
hardness ratios, defined as (H-S)/(H+S),  showing that spectral
evolution is present in some burst. }
\end{figure}
%%%%%%%%%%%%%%%%%%%%%%%%%%%%%%%%%

More violent bursts are the giant flares of SGRs, whose energy release are typically
$\sim(2-500)\times10^{44}$ ergs. There are three giant flares observed to date and
all of them are from SGRs~\cite{Mereghetti2008}. Giant flares are characterized by
a short hard spike followed by a longer pulsating tail (Fig. $4$ from ~\cite{Mereghetti2008}),
the peak luminosity of the initial spike can reach a few $10^{47}\ \rm{erg\ s^{-1}}$.
The rise time of the spike is smaller than a few milliseconds while the duration is
typically a few tenths of second. The spectrum of the spike are much harder than that
of the normal SGR short, with temperatures of hundreds of keV.
%but the evolution of the spectrum is still unclear due to limit of current observations.
The giant flares pulsating tails are characterized by a strong evolution of the flux,
timing and spectral properties. The decaying light curves, observed for a few minutes,
are strongly modulated at the neutron star rotation period, and show complex pulse
profiles which evolve with time. The spectra of the tails are well fitted by optically
thin bremsstrahlung models and are are softer than those of the initial spikes.
%
Besides bursts activities, long-term X-ray variability and transient phenomena
have also been discovered for AXPs and SGRs. Since AXPs and SGRs are not typical X-ray
pulsars which present variability due to accreting, the long-term X-ray variability and
transient phenomena would require the models to explain the large luminosity range and
thus of great importance.

%%%%%%%%%%%%%%%%%%%%%%%%%%%%%%%%%%%
\begin{figure}
\centering
\rotatebox{0}{\includegraphics[width=12cm]{GF_LC.eps}}
%\psfig{figure=GF_LC.eps,angle=0,width=13cm}
 \caption{Light
curves of the three giant flares from SGRs. Top panel: \lmc\
(Venera data in the 50-150 keV range, from \cite{Mazets82}); middle
panel: \zerozero\ (Ulysses data in the 20-150 keV range, courtesy
K. Hurley); bottom panel:  \zerosei\ (INTEGRAL SPI/ACS at E$>$80
keV, from \cite{Mere05d}). The initial peaks of the flares for
\lmc\ and \zerosei\ are out of the vertical scale. }
\end{figure}
%%%%%%%%%%%%%%%%%%%%%%%%%%%%%%%%%


%Long-term variations of AXPs can occur either as gradual changes
%in the flux, accompanied by variations in the spectrum, pulse profiles, and spin-down rate,
%or as sudden outbursts associated with energetic events occurring on short timescales,
%such as glitches and bursts~\cite{Mereghetti2008}. Transient phenomena which is corresponding
%to a larger luminosity variation have been discovered for several AXPs and one SGR, and
%suggest a connection between the bursting activity and the luminosity. Fig.~\ref{fig-xtej1810_lcurve}
%shows the X-ray light curve of the transient AXP \xte (from ~\cite{Mereghetti2008}).

%%%%%%%%%%%%%%%%%%%%%%%%%%%%%%%%%%%
%\begin{figure}
%\centering
%\rotatebox{-90}{\includegraphics[width=8cm]{j1810_flux_got07b.ps}}
%\psfig{figure=j1810_flux_got07b.ps,angle=-90,width=8.5cm}
%\caption{X-ray light curve of the outburst of the transient AXP
%\xte\ (from \cite{Gott07})} \label{fig-xtej1810_lcurve}
%\end{figure}
%%%%%%%%%%%%%%%%%%%%%%%%%%%%%%%%%

\subsubsection{Timing properties}

The spin periods of AXPs and SGRs distribute in a narrow range, $2-12$ s, and the period
derivatives are typically large compared to radio pulsars (Fig.~\ref{PPdot} from ~\cite{Tong}),
though people have discovered small period derivative SGR 0418$+$5729~\cite{Rea10}.
%
AXPs have a level of timing noise larger than that typically observed in radio pulsars, and
the timing noise is even larger in the SGRs.
%
Glitches have been observed in practically all the AXPs for which adequate timing data have
been taken over sufficiently long time periods. Most of the AXP glitch properties are consistent
with those of young radio pulsars, indicating that AXPs and SGRs are relatively young pulsars.
However, their glitch amplitude and frequency are larger than in radio pulsars of comparable
spin periods, which exhibit smaller and more rare glitches. The origin of glitches is still
unknown, but it seems that the variety of glitch properties in AXPs and SGRs can be better
explained in terms of star quakes models~\cite{Mereghetti2008}.
%
Another interesting discovery is the quasi periodic oscillations (QPOs) in the decaying
tails of SGRs giant flares. This phenomenon was first discovered in the RXTE data of the
energetic giant flare of SGR 1806$-$20 on December 27, 2004~\cite{Israel05} and was
independently confirmed with the data of RHESSI satellite~\cite{watts06}. In the RXTE
data of the giant flare of SGR 1900$+$14 in the August 1998, QPOs were also discovered~\cite{str05}.
The QPOs observed in the tails of giant flares are most likely due to seismic oscillations
induced by the large crustal fractures occurring in these extremely energetic events,
similar to what happens after earthquakes~\cite{Mereghetti2008}.


\begin{figure}
\label{PPdot}
\centering
 \includegraphics[width=12cm]{PPdot_PSRLikeObjects.eps}
\caption{P-$\mathrm{\dot{P}}$ diagram of pulsars. Squares are for
AXPs and SGRs, the six-pointed-star is for the radio loud magnetar,
the down-arrow marks the low magnetic field SGR (from McGill online
catalog:
http://www.physics.mcgill.ca/$\sim$pulsar/magnetar/main.html).
Diamonds are for X-ray dim isolated neutron stars (XDINSs) (from
Kaplan \& van Kerkwijk 2011). Stars are for rotating radio
transients (RRATs), dots are for normal and millisecond
pulsars (from ATNF:
http://www.atnf.csiro.au/research/pulsar/psrcat/). }
\end{figure}
%


\subsubsection{Magnetars VS ``quark star+fallback disk''}

A fundamental problem in AXP and SGR studies is to find a reliable power and to balance
the energy budget for both their persistent and burst emissions, since the persistent
X-ray luminosity of AXPs and SGRs is larger than their spin-down power and the luminosity
of recurrent bursts can reach as large as $10^{42}\ \rm{erg\ s^{-1}}$~\cite{Tong}. One
possible explanation is that isolated neutron stars are powered by magnetic energy, which
is called the magnetar model. In the magnetar model, AXPs and SGRs are powered by strong
magnetic fields.
%The extremely high magnetic fields, up to $3\times10^{17}\times(1\ \rm{ms}/P_{\rm{0}})$
%are formed through an efficient dynamo, supposing that the neutron stars are born with
%sufficiently small periods, of the order of $P_{\rm{0}}\sim1-2$ ms, and the convection is
%present~\cite{Thompson93}. Due to the strong magnetic dipole radiation losses, the spin
%down rate of magnetars is large, and the narrow distribution of spin period is explained
%as that young magnetars can spin down to several seconds in a few thousands years and
%will rapidly evolve toward the "death line". As spinning down, the magnetic energy,
%$E_{\rm{mag}}\sim10^{46}(P/5\ \rm{s})(\dot P/10^{-11}\ \rm{s\ s^{-1}})$ ergs,
%will be soon much larger than their rotational energy~\cite{Mereghetti2008}.
Magnetic field decay can provide a significant source of internal heating, and then produce
the persistent X-ray emission. The bursts and flares can be explained by magnetic
reconnections~\cite{Thompson93,Thompson95}.

Although the magnetar model is well established and widely used, there are accumulating
difficulties for it in recent observations.
%We did not find AXPs with large
%kick velocity~\cite{hel07} nor the corresponding energetic supernova, which are predicted
%in order to form the high magnetic fields~\cite{Gae01,vink06}. And we also did not detect
%the gamma-ray emissions from AXPs and SGRs~\cite{Tong11}.
%
Therefore, alternative modeling of AXPs and SGRs are not only possible but also very necessary~\cite{Tong}.
The ``quark star+fallback disk'' model could give explanations of different phenomena of
AXPs and SGRs. Some of the explosive material of the supernova may fall back onto the
neutron star and form a disk if the fallback material carries some amount of angular
momentum. The period clustering of AXPs and SGRs can naturally explained by the equilibrium
period which is reached when the corotation radius equals the magnetospheric radius,
and the presence of a disk is invoked to account for the rapid spin-down~\cite{Alpar,Chatterjee00}.
The persistent emission spectral of AXPs and SGRs can be modeled similarly to that of
accretion systems~\cite{Trumper}, and the bursts are due to sudden energy release of
the quark star, which may include elastic energy, gravitational energy, and conversion
energy from normal matter to quark matter~\cite{Tong}. The super-Eddington luminosity
bursts can also be understood by a bare quark star surface since quark matter are self-bond.
%In summary, both the bursts and persistent emissions of AXPs and SGRs are understandable
%in the ``quark star+fallback disk'' model. The period clustering and super-Eddington
%luminosity bursts are natural consequences of this model.


\subsubsection{Hard X-ray and radio emission from AXPs and SGRs}

Until a few years ago, the hard X-ray emissions from AXPs and SGRs were limited to the
bursts and flares. The detection of persistent hard X-ray emissions from AXPs extending
to $\sim150$ keV was surprising, considering their soft spectra below $10$ keV~\cite{Kuiper04,mol,den Hartog}.
The presence of hard X-ray tails in SGRs were also identified in earlier data~\cite{Mere05,Gotz}.
Up to now, emission above $20$ keV has been detected for five AXPs (4U 0142+61,
1RXS J170849.0-400910��1E 1048.59, 1E 1841-045 and 1E 1547.0.5408; e.g., \cite{den Hartog, Kuiper06, Leyder, Kuiper04, Enoto})
and two SGRs (SGR 1900+14, SGR 1806-20; e.g., \cite{Gotz, Esposito}). The upper limits
on the non-detected sources are not deep enough to exclude that they have similar hard
X-ray emission. The spectra of AXPs above $20$ keV are well fit with rather hard power
laws, while the spectra of SGRs are steeper (Fig.~\ref{fig-integral_spec}, from ~\cite{Mereghetti2008}).
Most importantly, the flat spectra imply that the energy released in the hard X-ray
range is a significant fraction of the total energy output from these sources~\cite{Mereghetti2008}.

%%%%%%%%%%%%%%%%%%%%%%%%%%%%%%%%%%%
\begin{figure}
\centering
\rotatebox{0}{\includegraphics[width=10cm]{int_magnetars.ps}}
%\psfig{figure=int_magnetars.ps,width=4cm}
\caption{XMM-Newton and INTEGRAL spectra of magnetars (from
\cite{Gotz}). Note the different behavior of SGRs (two top
panels) and AXPs: in the latter sources the spectra turn upward
above 10 keV, while in the SGRs the spectra steepen.}
\label{fig-integral_spec}
\end{figure}
%%%%%%%%%%%%%%%%%%%%%%%%%%%%%%%%%


Our knowledge of the radiation mechanism of hard X-ray emission from AXPs and SGRs
is still poor. On one the hand, current observations are limited by the low sensitivity
of the hard X-ray telescopes. On the other hand, theoretical spectral models are still
far from sophisticated.
%
In the context of the magnetar model, several mechanisms have been proposed. In Thompson
and Beloborodov (2005), two possibilities have been suggested, one is bremsstrahlung from
a thin turbulent layer of the star's surface, heated to $kT\sim100$ keV by magnetospheric
currents, and the other is synchrotron emission from pairs produced at a height of $\sim100$
km above the neutron star.
%In the first case a cut-off at a few hundred keV is expected,
%while in the second case the spectrum should extend to higher energies, peaking around $1$ MeV.
%
In Heyl and Hernquist (2005), the presence of a non-thermal distribution of electrons
and positrons in the outer parts of the magnetosphere were predicted according to the
fast-mode breakdown model, and the hard X-ray emission could instead be due to synchrotron
radiation of these changed particles.
%
In Baring and Harding (2007), the effect of resonant cyclotron scattering (RCS) is considered
in producing hard X-ray emission from magnetars,
%Considering that the plasma density in the magnetosphere of
%magnetars is high and in strong magnetic fields the Compton scattering is resonant at the
%cyclotron energy, with a cross section much higher than the Thomson value, the surface
%thermal photons ($kT\sim$ keV) propagating outward will be scattered resonantly to hard
%X-ray range and
and a flat spectrum is predicted (Fig.~\ref{fig:upscatt_spectra}, from ~\cite{Baring}).
%
In the fallback disk scenario, both the soft and the hard X-ray emission are explained as
the result of the accretion process. The power-law, hard X-ray spectra are produced in the
accretion flow mainly by bulk-motion Comptonization of soft photons emitted at the neutron
star surface. The hard and the soft X-ray spectra of AXP 4U 0142$+$61 are successfully reproduced
in Tr\"{u}mper et al.(2010).

%
\begin{figure}[h]
   \vspace{-7pt}
   \centering
   \includegraphics[width=0.7\textwidth]{reson_upscatt_spec.eps}
   \vspace{-8pt}
   \caption{Resonant Compton upscattering spectra (scaled) such as
   might be sampled in the magnetosphere of an AXP, for different
   relativistic electron Lorentz factors $\gamma_{e}$, as labelled.
   The emergent photon energy $\varepsilon_f$ is scaled in terms of
   $m_ec^2$.  The chosen magnetic field strengths of $B=3 B_{\rm cr}$
   (heavyweight, blue) and $B=0.3 B_{\rm cr}$ (lighter weight, red)
   correspond to different altitudes and perhaps colatitudes.
   Results are depicted for seed photons of energy $\varepsilon_s=0.003$
   (marked by the green vertical line), typical of thermal X-rays emanating
   from AXP surfaces; downscattering resonant emission at
   $\varepsilon_f < \varepsilon_s$ was not exhibited.
   }
 \label{fig:upscatt_spectra}
\end{figure}
%

Although detailed studies of the hard X-ray emission from AXPs and SGRs are hampered by the
relatively poor sensitivity of the current instruments, the situation will be different
in the near future as a few satellite missions (e.g., NuStar, HXMT) are now being developed
and expected to be operational soon. The significantly improved sensitivity will provide
much information about the radiation mechanism and it is promising to distinguish different
models through hard X-ray observations.

The pulsed radio emission from AXPs and SGRs was discovered until 2004. A point-like radio
source associated to the transient XTE J1810$-$197 was identified about $1$ year after the
X-ray outburst~\cite{hal05b}. The radio pulsations of another transient AXP 1E1547-54 was
also confirmed later~\cite{cam07b}. However, no radio pulsations were seen in the other
transient AXP, CXOU J1647$-$45, after its September 2006 outburst~\cite{Bur06}, nor in 1E
1048$-$59 after the flux enhancement accompanied by a glitch that occurred in March 2007~\cite{cam07c}.
Deep searches for radio pulsations in persistent AXPs have so far given negative results~\cite{Bur06a}.
%
The radio properties of the two AXPs showing radio pulsations differ in several respects
from those of radio pulsars: their flux is highly variable on daily timescales, their spectrum
is very flat, and their average pulse profile changes with time~\cite{cam08}. Such differences
probably indicate that the radio emitting regions are more complex than the dipolar open field
lines along which the radio emission in normal pulsars is thought to originate~\cite{Mereghetti2008}.
The origin of the radio emission is still controversial, although a few theoretical works have
started to address this issue~\cite{Bel09,Tho08}.
%It was initially believed that the absence
%of radio emission was a distinctive characteristic of magnetars, since the photon splitting
%in the high magnetic field could dominate over pair creation, thus suppressing the charged particle
%cascades that are at the origin of the radio emission~\cite{Baring01}. However, photon
%splitting applies only to one polarization mode: photons of the other mode cannot split.
%Therefore this argument does not apply.


\subsection{Microlensing pulsars}

Pulsar-like compact stars could be normal neutron stars or quark stars, and it should be
very important to affirm or negate the existence of either neutron or quark stars in order to
guide physicists in studying the nature of fundamental strong interaction.
%
With regard to the possible ways of identifying quark stars~\cite{Xu08},
it could be very straightforward and clear if we would find low-mass quark stars, since neutron
stars are essentially gravitation-bound while low-mass quark stars are mainly confined by strong interaction.
If we can detect a pulsar-like star with mass $\lesssim 0.1M_\odot$, then it is surely a quark star.
%

%
Up to now only the masses of compact stars in binary systems have been determined, by either Keplerian or
post-Keplerian parameters.
%
The lowest mass detected of an eclipsing X-ray pulsar, SMC X-1, is $1.06_{-0.10}^{+0.11}~M_\odot$, which
is near the minimum mass expected for a neutron star produced in a supernova~\cite{lmc}.
%
However, most of pulsars are isolated, and it is still a big challenge to measure their masses. It was
suggested to observationally determine an isolated neutron star's mass from the red-shift (as a function
of the ratio of mass to radius, $M/R$) and pressure broadening (as a function of $M/R^2$) of an absorption
spectrum, yet no atomic line has been detected with certainty in the thermal X-ray spectra~\cite{Xu02}.

In our work in 2010~\cite{Dai10}, we proposed to determine the mass of isolated neutron stars through
gravitational microlensing. We estimated the lensing rates of pulsars towards the galactic bulge and
spiral arms and showed that, for FAST and SKA, the lensing rates ($\propto N$) are $\geq1$ event/decade
at least.
%
Therefore, it is hopeful to measure the mass of an isolated pulsar in the future with the method of
microlensing. We proposed to do catalogue comparison and microlensing prediction for pulsars identified
by SKA and FAST in the future, and to carry out microlensing observation coupled with radio observation in
order to detect microlensing pulsar events and to measure the masses of isolated pulsars with advanced
optical facilities.

\section{Research experience}

We investigated the possibilities that pulsars act as the lens in gravitational microlensing
events towards the galactic bulge or a spiral arm in 2010~\cite{Dai10}. Our estimation was
based on expectant survey and observations of FAST (Five hundred meter Aperture Spherical Telescope)
and SKA (Square Kilometer Array), and two different models of pulsar distribution were
used. We found that the lensing rate was $> 1$ event/decade, being high enough to search
the real events. Therefore, the microlensing observations focusing on pulsars identified
by FAST or SKA in the future are meaningful. As an independent determination
of pulsar mass, a future detection of microlensing pulsars should be significant in
the history of studying pulsars, especially in constraining the state of matter (either
hadronic or quark matter) at supra-nuclear densities. The observations of such events
by using advanced optical facilities (e.g., the James Webb Space Telescope and the
Thirty Meter Telescope) in future are highly suggested.
%

Recently, we compared the positions of known pulsars and those of microlensing
events and background stars, aiming to identify possible microlensing pulsar
events and predict microlensing pulsars candidates. For pulsars, we get their
positions from the ATNF Pulsar Catalogue. For microlensing events, we get the positions
from the OGLE and MACHO catalogues. For background stars, we consider the SDSS and
UKIDDS latest data release and check if there were any background star near each pulsar.
Our comparison showed that no known microlensing events were due to pulsars and no
photometric microlensing pulsar candidates were found. However, we showed that
astrometric microlensing events due to pulsars were expected in the future and the
amplitude of such events were strong enough to observe by the next generation of
telescopes such as GAIA. Therefore, with advanced telescopes, gravitational microlensing
would be a powerful method to measure the mass of isolated neutron stars in the future.
%

Another interesting thing is that we found a SDSS object close to the millisecond PSR J2317+1439.
PSR J2317$+$1439 is in a binary system. Its companion has not been identified yet, and previous
observations only set a limit of $R>24$~\cite{kerk04}. However, taking $0.1\ \rm{arcsec}$ as the
astrometric uncertainty, the optical position of this object ($\alpha_{\rm{2000}}=23^{\rm{h}}17^{\rm{m}}09^{\rm{s}}.236$,
$\delta_{\rm{2000}}=+14^{\circ}39'31''.273$) is offset from the timing
position of the pulsar ($\alpha_{\rm{2000}}=23^{\rm{h}}17^{\rm{m}}09^{\rm{s}}.23723(7)$,
$\delta_{\rm{2000}}=+14^{\circ}39'31''.220(2)$ (Camilo et al. 1996)) by $0.05\pm0.1\ \rm{arcsec}$
(see Fig.~\ref{photo}). We fitted the observed absolute magnitude and colors against the
predictions from the helium-core white dwarf models and showed that the magnitude and colors
are compatible with cooling tracks and such a white dwarf might be rather hot, $T_{\rm{eff}}\sim8500\ \rm{K}$ (see Fig.~\ref{color}).
Such a hot white dwarf would challenge our current understanding of the properties of companions
to millisecond pulsars and the evolution of the system. The possible brightness variation of
this object is also interesting. Therefore, we have proposed to use the MegaCam of CFHT
to take g-band, r-band, and i-band images of the field containing PSR J2317$+$1439.
%
%%%%%%%%%%%%%%%%%%%%%%%%%%%%%%%%%%%%%%%%%
\begin{figure*}
\begin{center}
  %
  \includegraphics[width=5.5 in]{3fig.ps}
%
\caption{$50''\times60''$ subsection of the g-band, r-band and i-band SDSS images of the field containing
PSR J2317$-$1439. The $2''$ tick marks indicate the position of PSR J2317$-$1439. }
\end{center}
\label{photo}
\end{figure*}
%%%%%%%%%%%%%%%%%%%%%%%%%%%%%%%%%%%%%%%%%%

%
%%%%%%%%%%%%%%%%%%%%%%%%%%%%%%%%%%%%%%%%%
\begin{figure*}
\begin{center}
  %
  \includegraphics[width=5 in]{HR.eps}
%
\caption{Color-magnitude diagrams of the counterpart to PSR J2317$-$1439. The colors and magnitudes are
indicated with error bars. Also shown are helium-core white dwarf cooling tracks (solid lines) by
Serenelli et al. (2001). The mass of different cooling models and the cooling age of white dwarfs are
indicated.}
\end{center}
\label{color}
\end{figure*}
%%%%%%%%%%%%%%%%%%%%%%%%%%%%%%%%%%%%%%%%%%
%
%

We have also done some research on the hard X-ray emission of AXPs and SGRs in order to
investigate the feasibility of using HXMT to distinguish different models. With the redistribution
matrix file and the auxiliary response file provided by the HXMT Science Definition Team, we
simulated the spectra of AXP 4U 0142$+$61 with both the RCS and BMC models, our results show
that observations of HXMT can distinguish these two models. More details will be presented
later.


\section{Research plan}

Diverse and energetic phenomena of pulsar-like compact stars make them one of the most
interesting objects in astrophysics and attract increasing attention. The understanding of
the energy sources and radiation mechanism is closely related to the equation of state,
evolution and magnetic fields of the neutron star and thus of great importance in fundamental
physics and astrophysics.
%
During my PhD, I will carry out multiband study of pulsar-like compact stars, aiming
to understand several interesting phenomena of pulsar-like compact stars and investigate the
nature of the neutron star.

The main part of my study will be focused on the radio and hard X-ray emission and timing
properties of AXPs and SGRs.
%After more than $10$ years of extensive multiband observations, a good harvest of data of
%AXPs and SGRs have been accumulated, and people have propose different models to explain
%diverse phenomena. However, the nature of AXPs and SGRs is still controversial, and it is
%difficult to get large advances in the classical X-ray band, due to the lack of new missions
%significantly improving the capabilities of the currently available big observatories like
%XMM-Newton and Chandra. The situation is more promising for the hard X-ray band, since the
%presence of hard X-ray emission from most magnetars has been established, and a few satellite
%missions with significantly improved sensitivity (e.g., NuStar, HXMT) are now being developed
%and expected to be operational in the near future. Also, the detection of pulsed radio emission
%from the AXP XTE J1810$-$197~\cite{Camilo} opened a new perspective in the study of AXPs and
%SGRs.
%
%
%My research in the next few years will be focused on the multiband radiation and timing properties of AXPs and SGRs,
%aiming to understand different phenomena and distinguish different models based
%and hence study the equation of state, evolution and magnetic field of the neutron star.
Both theoretical and observational research will be carried out. For the theoretical part, I will
study the radiation mechanism of the radio and X-ray emission and the origin of glitches and QPOs.
%
As for the radio emission, no complete theory has been put forward so far, although a few
theoretical works have started to address this issue~\cite{Bel09,Tho08}. According to the delay in
the appearance of the radio emission after the X-ray outburst onset, it seems that the radio emission
is connected to burst activities. In Zanotti et al.(2011), the effects of toroidal oscillations at
the star surface have been considered in particle acceleration processes in the polar cap region of an
oscillating neutron star, and their results show that significant enhancements of the Lorentz factor
of electrons are produced by star oscillations. Considering that burst activities might induce star
oscillations, we propose to explain the radio emission after outburst as the effects of star oscillations
and will carry out studies in the context of quark stars.
%
As for the X-ray emission, the radiation mechanism will be investigated in the fallback disk model.
In Tr\"{u}mper et al.(2010), they first applied the bulk-motion Comptonization (BMC) of soft photons
emitted at the neutron star surface by the accretion flow to understand the X-ray emission from
AXPs and SGRs, and both the soft and the hard X-ray spectra of AXP 4U 0142$+$61 were well fitted (Fig.~\ref{Fig:Spec1} from ~\cite{Trumper}).
Considering that AXP 4U 0142+61 is the first AXP with direct observational support for the presence
of fallback disks around these systems~\cite{Wang}, their work indicates that the effect of bulk-motion
Comptonization might be important in the hard X-ray radiation process. However, the BMC model
are currently only applied to one AXP, and it seems hard to fit both the soft and the hard X-ray
spectra of some AXPs and SGRs. Therefore, we suppose that there are other components accounting
for the hard X-ray emission from AXPs and SGRs besides the BMC component. One promising mechanism is
the resonant cyclotron scattering, which has been applied to normal pulsars and magnetars to explain
the hard X-ray emission. We plan to take the effects of resonant cyclotron scattering into consideration,
and predict a two components spectral model for the hard X-ray emission of AXPs and SGRs under
the fallback disk scenario.
%
I will also investigate the origin of glitches and QPOs in the context of solid quark star. The variety
of glitch properties in AXPs and SGRs indicate that their origin might be corresponding to star quakes,
and QPOs are also likely due to seismic oscillations induced by the large crustal fractures occurring
in these extremely energetic events. These quakes and oscillations of the star can be naturally understood
by the solid state of quark cluster star~\cite{Tong11} and I will carry out further study.

\begin{figure}
\centering
\includegraphics[angle=-90,width=10cm]{fig1.ps}
\caption{Top panel: The {\textit{Chandra}} HEG and MEG (first positive
order shown with red and green points respectively) and INTEGRAL ISGRI
data (shown in black) along with the best-fit {\sc{comptb}} model.
Bottom panel: Ratio between the data and the best-fit model.}
\label{Fig:Spec1}
\end{figure}

For the observational part, I will try to investigate and distinguish different spectral models,
based on archived data and future observations. The Immediate goal is applying our spectral model to
understand the soft and hard X-ray emission of known AXPs and SGRs, using archived data of Chandra,
XMM-Newton and INTEGRAL. Although it is still hard to distinguish different models due to quality
of current data, such works would verify that both the soft and hard X-ray emission can be naturally
explained in the context of fallback disk model.
%
On the other hand, a few satellite missions with significantly improved sensitivity are now being
developed and expected to be operational in the near future. Especially, the Hard X-ray Modulation
Telescope (HXMT), which is developed by China, will play an important role in AXPs and SGRs study
with its wide energy range and high sensitivity in hard X-ray band.
%
We have investigated the feasibility of using HXMT to distinguish different spectral models of the
hard X-ray emission of AXPs and SGRs. We mainly considered the RCS model of magnetar and the BMC model
of fallback disk system, since they are well established and represent different understanding of AXPs
and SGRs. The spectra of these two models show obvious difference in the hard X-ray band. The RCS model
predicts a flat spectra extending to MeV~\cite{Baring}, while the spectrum of the BMC model turn off at $\sim200$ keV.
With the redistribution matrix file and the auxiliary response file provided by the HXMT Science
Definition Team, we simulated the spectra of AXP 4U 0142$+$61 with both the RCS and BMC models, our
results show that observations of HXMT can distinguish these two models. Fig.~\ref{figure 1} shows
the simulated spectra of AXP 4U 0142$+$61 with the RCS and BMC models, the exposure time is $10$ Ms.
Fig.~\ref{figure 2} and Fig.~\ref{figure 3} show the results of fitting the simulated BMC spectrum
with RCS model and BMC model. Fig.~\ref{figure 4} to Fig.~\ref{figure 6} show the results with exposure
time of $1$ Ms. Fig.~\ref{figure 7} shows the simulated $1-250$ keV spectrum of AXP 4U 0142$+$61 with
the BMC model.
%
Observations with HXMT will also significantly improve the timing quality of AXPs and SGRs. Since
glitches have been observed in practically all the AXPs for which adequate timing data have been taken
over sufficiently long time periods, HXMT will discover much more glitches and promote our understanding
of the origin of glitches.

%%%%%%%%%%%%%%%%%%%%%%%%%%%%%%%%%%%%%%%%%%%%%%%%%%%%%%%%%%%%%%%%%%
\begin{figure}[h]
\begin{center}
  % Requires \usepackage{graphicx}
  \rotatebox{-90}{\includegraphics[width=8cm]{10ms_comptb_po.ps}}
\end{center}
\caption{Simulated spectra with exposure time of $10$ Ms. Red line and points represent the spectrum of RCS model.
Black line and points represent the spectrum of BMC model.}
\label{figure 1}
\end{figure}
%%%%%%%%%%%%%%%%%%%%%%%%%%%%%%%%%%%%%%%

%%%%%%%%%%%%%%%%%%%%%%%%%%%%%%%%%%%%%%%%%%%%%%%%%%%%%%%%%%%%%%%%%%
\begin{figure}[h]
\begin{center}
  % Requires \usepackage{graphicx}
  \rotatebox{-90}{\includegraphics[width=8cm]{10ms_comptb_fit_po.ps}}
  % Requires \usepackage{graphicx}
\end{center}
\caption{Fitting the simulated BMC spectrum with the RCS model.}
\label{figure 2}
\end{figure}
%%%%%%%%%%%%%%%%%%%%%%%%%%%%%%%%%%%%%%%

%%%%%%%%%%%%%%%%%%%%%%%%%%%%%%%%%%%%%%%%%%%%%%%%%%%%%%%%%%%%%%%%%%
\begin{figure}[h]
\begin{center}
  % Requires \usepackage{graphicx}
  \rotatebox{-90}{\includegraphics[width=8cm]{10ms_comptb_fit_comptb.ps}}
  % Requires \usepackage{graphicx}
\end{center}
\caption{Fitting the simulated BMC spectrum with the BMC model as a comparison with Fig.~\ref{figure 3}.}
\label{figure 3}
\end{figure}
%%%%%%%%%%%%%%%%%%%%%%%%%%%%%%%%%%%%%%%

%%%%%%%%%%%%%%%%%%%%%%%%%%%%%%%%%%%%%%%%%%%%%%%%%%%%%%%%%%%%%%%%%
\begin{figure}[h]
\begin{center}
  % Requires \usepackage{graphicx}
  \rotatebox{-90}{\includegraphics[width=8cm]{1ms_comptb_po.ps}}
\end{center}
\caption{Same as Fig.~\ref{figure 1}��with exposure time of $1$ Ms.}
\label{figure 4}
\end{figure}
%%%%%%%%%%%%%%%%%%%%%%%%%%%%%%%%%%%%%%%

%%%%%%%%%%%%%%%%%%%%%%%%%%%%%%%%%%%%%%%%%%%%%%%%%%%%%%%%%%%%%%%%%%
\begin{figure}[h]
\begin{center}
  % Requires \usepackage{graphicx}
  \rotatebox{-90}{\includegraphics[width=8cm]{1ms_comptb_fit_po.ps}}
  % Requires \usepackage{graphicx}
\end{center}
\caption{Same as Fig.~\ref{figure 2}��with exposure time of $1$ Ms.}
\label{figure 5}
\end{figure}
%%%%%%%%%%%%%%%%%%%%%%%%%%%%%%%%%%%%%%%

%%%%%%%%%%%%%%%%%%%%%%%%%%%%%%%%%%%%%%%%%%%%%%%%%%%%%%%%%%%%%%%%%%
\begin{figure}[h]
\begin{center}
  % Requires \usepackage{graphicx}
  \rotatebox{-90}{\includegraphics[width=8cm]{1ms_comptb_fit_comptb.ps}}
  % Requires \usepackage{graphicx}
\end{center}
\caption{Same as Fig.~\ref{figure 3}��with exposure time of $1$ Ms.}
\label{figure 6}
\end{figure}
%%%%%%%%%%%%%%%%%%%%%%%%%%%%%%%%%%%%%%%

%%%%%%%%%%%%%%%%%%%%%%%%%%%%%%%%%%%%%%%%%%%%%%%%%%%%%%%%%%%%%%%%%%
\begin{figure}[h]
\begin{center}
  % Requires \usepackage{graphicx}
  \rotatebox{-90}{\includegraphics[width=8cm]{0142_whole_comptb.ps}}
  % Requires \usepackage{graphicx}
\end{center}
\caption{Simulated $1-250$ keV spectrum of AXP 4U 0142$+$61 with the BMC model, with exposure time of $1$ Ms.}
\label{figure 7}
\end{figure}
%%%%%%%%%%%%%%%%%%%%%%%%%%%%%%%%%%%%%%%

Besides the research of AXPs and SGRs, I will continue my study on microlensing pulsars. In our work
in 2010, we have investigated the possibilities that pulsars act as the lens in gravitational microlensing
events towards the galactic bulge or a spiral arm~\cite{Dai10}. Our estimation was based on
expectant survey and observations of FAST and SKA, and it showed that the lensing rate was $> 1$ event/decade,
being high enough to search the real events. Therefore, the microlensing observations focusing on pulsars
identified by FAST or SKA in the future are meaningful. Recently, we compared the positions of known
pulsars and those of microlensing events and background stars, aiming to identify possible microlensing
pulsar events and predict microlensing pulsars candidates. Our comparison showed that no known microlensing
events were due to pulsars and no photometric microlensing pulsar candidates were found. However, we
showed that astrometric microlensing events due to pulsars were expected in the future and the amplitude
of such events were strong enough to observe by the next generation of telescopes such as GAIA.
We are now preparing the paper.
%

Another interesting discover of our catalogue comparison is that we found a SDSS object close to the
millisecond PSR J2317+1439. PSR J2317$+$1439 is in a binary system. Its companion has not been identified yet,
and previous observations only set a limit of $R>24$~\cite{kerk04}. However, taking $0.1\ \rm{arcsec}$
as the astrometric uncertainty, the optical position of this object ($\alpha_{\rm{2000}}=23^{\rm{h}}17^{\rm{m}}09^{\rm{s}}.236$,
$\delta_{\rm{2000}}=+14^{\circ}39'31''.273$) is offset from the timing
position of the pulsar ($\alpha_{\rm{2000}}=23^{\rm{h}}17^{\rm{m}}09^{\rm{s}}.23723(7)$,
$\delta_{\rm{2000}}=+14^{\circ}39'31''.220(2)$ (Camilo et al. 1996)) by
$0.05\pm0.1\ \rm{arcsec}$. We fitted the observed absolute magnitude and colors against the
predictions from the helium-core white dwarf models and showed that the magnitude and colors
are compatible with cooling tracks and such a white dwarf might be rather hot, $T_{\rm{eff}}\sim8500\ \rm{K}$.
Such a hot white dwarf would challenge our current understanding of the properties of companions
to millisecond pulsars and the evolution of the system. The possible brightness variation of
this object is also interesting. Therefore, we have submitted a proposal to use the MegaCam of CFHT
to take g-band, r-band, and i-band images of the field containing PSR J2317$+$1439. If our proposal succeeded,
we will have chance to identify the companion to PSR J2317$+$1439 and investigate the evolution of
this system.

I will also participate radio observations of pulsars which will be carried out with the $40\ \rm{m}$
radio telescope of Yunnan Observatory. Regular timing of normal pulsars, radio emission of AXPs and
SGRs, timing of high-magnetic field pulsars and searching pulsars around the galactic center will
be possible science goals. It is also a good opportunity to learn pulsar timing and searching.

Finally, a brief time schedule is as fellow:

2012--2013: Finish current microlensing study and write a paper; If our proposal succeeded, analyze the data
and do research; Investigate the radiation mechanism under the fallback disk scenario;
Learn data analysis and X-ray timing of pulsars.

2013--2014: Apply our spectral model to investigate the spectra of AXPs and SGRs with archived data. Learn
data analysis and X-ray timing of pulsars; Do radio pulsar timing and searching.

2014--2015: Analyze data of HXMT, and investigate the timing and spectral properties of AXPs and SGRs;
Do radio pulsar timing and searching.

%\begin{equation}
%\left[{\begin{array}{c}
%\sin\theta\cos\phi \\
%\sin\theta\sin\phi \\
%\cos\theta \\
%\end{array}}\right]=\left[{\begin{array}{ccc}
%1 & 0 & 0 \\
%0 & \cos\theta_0 & \sin\theta_0 \\
%0 & -\sin\theta_0 & \cos\theta_0 \\
%\end{array}}\right]\left[{\begin{array}{ccc}
%\cos\phi_0 & \sin\phi_0 & 0 \\
%-\sin\phi_0 & \cos\phi_0 & 0 \\
%0 & 0 & 1 \\
%\end{array}}\right]\left[{\begin{array}{c}
%\sin\vartheta\cos\varphi \\
%\sin\vartheta\sin\varphi \\
%\cos\vartheta \\
%\end{array}}\right]
%\end{equation}



\begin{thebibliography}{99}

\bibitem[Abdo et al.(2011)]{Abdo}
Abdo A. A., et al. 2011, Science, 331, 739

\bibitem[Alcock et al.(1986a)]{Alcock86a}
Alcock C., Farhi E., Olinto A., 1986, ApJ, 310, 261

\bibitem[Alcock et al.(1986b)]{Alcock86b}
Alcock C., Farhi E., Olinto A., 1986, PRL, 57, 2088

\bibitem[Alpar(2001)]{Alpar}
Alpar M. A., 2001, ApJ, 554, 1245

\bibitem[Baring \& Harding(2001)]{Baring01}
Baring M. G., Harding A. K., 2001, ApJ, 547, 929

\bibitem[Baring \& Harding(2007)]{Baring}
Baring M. G., Harding, A. K., 2007, Ap\&SS, 308, 109

\bibitem[Beloborodov \& Thompson(2007)]{Beloborodov}
Beloborodov A. M., Thompson C., 2007, ApJ, 657, 967

\bibitem[Beloborodov(2007)]{Bel09}
Beloborodov A. M. 2009, ApJ, 703, 1044

\bibitem[Burgay et al.(2006a)]{Bur06a}
Burgay M., Rea N., Israel G. L., Possenti A., Burderi L., di Salvo T., D��Amico N., Stella L., 2006, MNRAS, 372, 410

\bibitem[Burgay et al.(2006b)]{Bur06}
Burgay M., Rea N., Israel G., Possenti A., 2006, Astron Telegr 903, 1

\bibitem[Camilo et al.(2006)]{Camilo}
Camilo F., Ransom S. M., Halpern J. P., Reynolds J., Helfand D. J., Zimmerman N., Sarkissian J., 2006, Nature, 442, 892

\bibitem[Camilo et al.(2007a)]{cam07}
Camilo F., Ransom S. M., Halpern J. P., Reynolds J., 2007, ApJ, 666, 93

\bibitem[Camilo et al.(2007b)]{cam07b}
Camilo F., Ransom S. M., Halpern J. P., Reynolds J., 2007, ApJ, 666, 93

\bibitem[Camilo et al.(2007c)]{cam07c}
Camilo F., Reynolds J., 2007, Astron Telegr, 1056, 1

\bibitem[Camilo et al.(2008)]{cam08}
Camilo F., Reynolds J., Johnston S., Halpern J. P., Ransom S. M., 2008, ApJ, 679, 681

\bibitem[Chatterjee et al.(2000)]{Chatterjee00}
Chatterjee P., Hernquist L., Narayan R., 2000, ApJ, 534, 373

\bibitem[Cline et al.(1980)]{cli80}
Cline T. L., Desai U. D., Pizzichini G., et~al. 1980, ApJ, 237, 1

\bibitem[Corbel et al.(1999)]{cor99}
Corbel S., Chapuis C., Dame T. M., Durouchoux P., 1999, ApJ, 526, 29

\bibitem[Corbel \& Eikenberry(2004)]{cor04}
Corbel S., Eikenberry S. S., 2004, A\&A, 419, 191

\bibitem[den Hartog et al.(2006)]{den Hartog}
den Hartog P. R., Hermsen W., Kuiper L., Vink J., in��t Zand J. J. M., Collmar W., 2006, A\&A, 451, 587

\bibitem[Dai et al.(2010)]{Dai10}
Dai S., Xu R. X., Esamdin A., 2010, MNRAS, 405, 2754

\bibitem[Duncan \& Thompson(1992)]{Duncan}
Duncan R. C., Thompson C., 1992, ApJ, 392, 9

\bibitem[Durant \& van Kerkwijk(2006)]{dur06}
Durant M., van Kerkwijk M. H., 2006, ApJ, 650, 1070

\bibitem[Durant(2011)]{Durant}
Durant M., Kargaltsev O., Pavlov G. G., Kowalski P. M., Posselt B., van Kerkwijk M. H., Kaplan D. L., 2011, To appear in ApJ(arXiv:1111.2346)

\bibitem[Enoto et al.(2010)]{Enoto}	
Enoto T., Nakazawa K., Makishima K., Nakagawa Y. E., Sakamoto T., Ohno M., Takahashi T., Terada Y., Yamaoka K., Murakami T., Takahashi H., 2010, PASJ, 62, 475

\bibitem[Esposito et al.(2007a)]{Esposito07a}
Esposito P., Mereghetti S., Tiengo A., Sidoli L., Feroci M., Woods P., 2007, A\&A, 461, 605

\bibitem[Esposito et al.(2007b)]{Esposito}
Esposito P., Mereghetti S., Tiengo A., Zane S., Turolla R., G\"{o}tz D., Rea N., Kawai N., Ueno M., Israel G. L., Stella L., Feroci M., 2007, A\&A, 476, 321

\bibitem[Fahlman \& Gregory(1981)]{Fahlman}
Fahlman G. G., Gregory P. C., 1981, Nature, 293, 202

\bibitem[Farinelli(2008)]{Farinelli}
Farinelli R., Titarchuk L., Paizis A., Frontera F., 2008, ApJ, 680, 602

\bibitem[Fenimore et al.(1994)]{Feni94}
Fenimore E. E., Laros J. G., Ulmer A., 1994, ApJ, 432, 742

\bibitem[Feroci et al.(2004)]{Feroci}
Feroci M., Caliandro G. A., Massaro E., Mereghetti S., Woods P. M., 2004, ApJ, 612, 408

%\bibitem[Gaensler et al.(2001)]{Gae01}
%Gaensler B. M., Slane P. O., Gotthelf E. V., Vasisht G., 2001, ApJ, 559, 963

\bibitem[Gavriil et al.(2002)]{Gavriil}
Gavriil F. P., Kaspi V. M., Woods P. M., 2002, Nature, 419, 142

\bibitem[Gelfand \& Gaensler(2007)]{gel07}
Gelfand J. D., Gaensler B. M., 2007, ApJ, 667, 1111

\bibitem[Gotthelf \& Halpern(2007)]{Gott07}
Gotthelf E. V., Halpern J. P., 2007, Ap\&SS, 308, 79

\bibitem[G\"{o}tz et al.(2004)]{Gotz04}
G\"{o}tz D., Mereghetti S., Mirabel I. F., Hurley K., 2004, A\&A, 417, 45

\bibitem[G\"{o}tz et al.(2006)]{Gotz}
G\"{o}tz D., Mereghetti S., Tiengo A., Esposito P., 2006, A\&A, 449, 31

\bibitem[Halpern \& Gotthelf(2005a)]{hal05}
Halpern J. P., Gotthelf E. V., 2005, ApJ, 618, 874

\bibitem[Halpern et al.(2005b)]{hal05b}
Halpern J. P., Gotthelf E. V., Becker R. H., Helfand D. J., White R. L., 2005, ApJ, 632, 29

\bibitem[Halpern et al.(2008)]{hal08}
Halpern J. P., Gotthelf E. V., Reynolds J., Ransom S. M., Camilo F., 2008, ApJ, 676, 1178

%\bibitem[Helfand et al.(2007)]{hel07}
%Helfand D. J., et al., 2007, ApJ, 662, 1198

\bibitem[Heyl \& Hernquist(2005)]{Heyl}
Heyl J. S., Hernquist L., 2005, ApJ, 618, 463

\bibitem[Hurley et al.(1994)]{hur94}
Hurley K. J., McBreen B., Rabbette M., Steel S., 1994, A\&A, 288, 49

\bibitem[Hurley et al.(1999)]{hur99e}
Hurley K., Li P., Kouveliotou C., et al. 1999, ApJ, 510, 111

\bibitem[Ibrahim et al.(2004)]{ibr04}
Ibrahim A. I., Markwardt C. B., Swank J. H., et al. 2004, ApJ, 609, 21

\bibitem[Israel et al.(1994)]{Israel}
Israel G. L., Mereghetti S., Stella L., 1994, ApJ, 433, 28

\bibitem[Israel et al.(2005)]{Israel05}
Israel G. L., Belloni T., Stella L., Rephaeli Y., Gruber D. E., Casella P., Dall��Osso S., Rea N., Persic M., Rothschild R. E., 2005, ApJ,
628, 53

\bibitem[Israel et al.(2007)]{isr07}
Israel G. L., Campana S., Dall'Osso S., Muno M. P., Cummings J., Perna R., Stella L., 2007, ApJ, 664, 448

\bibitem[Kaplan \& van Kerkwijk(2011)]{Kaplan11}
Kaplan D. L., van Kerkwijk M. H., 2011, arXiv:1109.2105.

\bibitem[Kaspi et al.(2000)]{Kaspi00}
Kaspi V. M., Lackey J. R., Chakrabarty D., 2000, ApJ, 537, 31

\bibitem[Kaspi et al.(2003)]{Kaspi2003}
Kaspi V. M., Gavriil F. P., Woods P. M., Jensen J. B., Roberts M. S. E., Chakrabarty D., 2003, ApJ, 588, 93

\bibitem[Kaspi \& Boydstun(2010)]{Kaspi2010}
Kaspi V. M., Boydstun K., 2010, ApJ, 710, 115

\bibitem[Kothes \& Dougherty(2007)]{kot07}
Kothes R., Dougherty S. M., 2007, A\&A, 468, 993

\bibitem[Kouveliotou et al.(1998)]{Kouveliotou}
Kouveliotou C., Dieters S., Strohmayer T., van Paradijs J., Fishman G. J., Meegan C. A., Hurley K., Kommers J., Smith I., Frail D., Murakami T., 1998, Nature, 393, 235

\bibitem[Kuiper et al.(2004)]{Kuiper04}
Kuiper L., Hermsen W., Mendez M., 2004, ApJ, 613, 1173

\bibitem[Kuiper et al.(2006)]{Kuiper06}
Kuiper L., Hermsen W., den Hartog P. R., Collmar W., 2006, ApJ, 645, 556

\bibitem[Lamb et al.(2002)]{lam02}
Lamb R. C., Fox D. W., Macomb D. J., Prince T. A. 2002, ApJ, 574, 29

\bibitem[Laros et al.(1986)]{lar86}
Laros J. G., Fenimore E. E., Fikani M. M., Klebesadel R. W., Barat C., 1986, Nature, 322, 152

\bibitem[Leyder et al.(2008)]{Leyder}
Leyder J. C., Walter R., Rauw G., 2008, A\&A, 477, 29

\bibitem[Mazets et al.(1979a)]{Mazets79a}
Mazets E. P., Golenetskij S. V., Guryan Y. A., 1979a, Sov. Astron. Lett., 5, 343

\bibitem[Mazets et al.(1979b)]{Mazets79b}
Mazets E. P., Golentskii S. V., Ilinskii V. N., Aptekar R. L., Guryan I. A., 1979b, Nature, 282, 587

\bibitem[Mazets et al.(1982)]{Mazets82}
Mazets E. P., Golenetskii S. V., Gurian I. A., Ilinskii V. N., 1982, Ap\&SS, 84, 173

\bibitem[Mereghetti \& Stella(1995)]{mer95}
Mereghetti S., Stella L. 1995, ApJ, 442, 17

\bibitem[Mereghetti et al.(2005a)]{Mere05}
Mereghetti S., G\"{o}tz D., Mirabel I. F., Hurley K., 2005, A\&A, 433, 9

\bibitem[Mereghetti et al.(2005b)]{Mere05b}
Mereghetti S., Tiengo A., Esposito P., G\"{o}tz D., Stella L., Israel G. L., Rea N., Feroci M., Turolla R., Zane S., 2005, ApJ, 628, 938

\bibitem[Mereghetti et al.(2005c)]{Mere05c}
Mereghetti S., G\"{o}tz D., Mirabel I. F., Hurley K., 2005, A\&A, 433, 9

\bibitem[Mereghetti et al.(2005d)]{Mere05d}
Mereghetti S., G\"{o}tz D., Kienlin Avon, Rau A., Lichti G., Weidenspointner G., Jean P., 2005, ApJ, 624, 105

\bibitem[Mereghetti et al.(2006)]{Mere06}
Mereghetti S., Esposito P., Tiengo A., Zane S., Turolla R., Stella L., Israel G. L., G\"{o}tz D., Feroci M., 2006, ApJ, 653, 1423

\bibitem[Mereghetti(2008)]{Mereghetti2008}
Mereghetti S., 2008, Astron. Astrophys. Rev., 15, 225

\bibitem[Molkov et al.(2004)]{mol}
Molkov S. V., Cherepashchuk A. M., Lutovinov A. A., Revnivtsev M. G., Postnov K. A., Sunyaev R. A., 2004, Sov. Astron. Lett., 30, 534

\bibitem[Muno et al.(2006)]{mun06}
Muno M. P., Clark J. S., Crowther P. A., et al. 2006, ApJ, 636, 41

\bibitem[Muno et al.(2007)]{mun07}
Muno M. P., Gaensler B. M., Clark J. S., et al. 2007, MNRAS, 378, 44

\bibitem[Paczy\'{n}ski(1992)]{Paczynski96}
Paczy\'{n}ski B., 1992, ACTA ASTRONOMICA, 42, 145

\bibitem[Rea et al.(2010)]{Rea10}
Rea N., Esposito P., Turolla R., Israel G. L., et al., 2010, Science, 330, 944

\bibitem[Rothschild et al.(1994)]{rot94}
Rothschild R. E., Kulkarni S. R., Lingenfelter R. E., 1994, Nature, 368, 432

\bibitem[Seward et al.(1986)]{Seward}
Seward F. D., Charles P. A., Smale A. P., 1986, ApJ, 305, 814

\bibitem[Strohmayer \& Watts(2005)]{str05}
Strohmayer T. E., Watts A. L., 2005, ApJ, 632, 111

\bibitem[Sugizaki et al.(1997)]{sug97}
Sugizaki M., Nagase F., Torii K., et al. 1997, PASJ, 49, 25

\bibitem[Thompson \& Duncan(1992)]{Thompson92}
Thompson C., Duncan R. C., 1992, ApJ, 392, 9

\bibitem[Thompson \& Duncan(1993)]{Thompson93}
Thompson C., Duncan R. C., 1993, ApJ, 408, 194

\bibitem[Thompson \& Duncan(1995)]{Thompson95}
Thompson C., Duncan R. C., 1995, MNRAS, 275, 255

\bibitem[Thompson \& Duncan(1996)]{Thompson96}
Thompson C., Duncan R. C., 1996, ApJ, 473, 322

\bibitem[Thompson et al.(2002)]{Thompson02}
Thompson C., Lyutikov M., Kulkarni S. R., 2002, ApJ, 574, 332

\bibitem[Thompson(2008)]{Tho08}
Thompson C., 2008, ApJ, 688, 499

\bibitem[Tian \& Leahy(2008)]{Tian08}
Tian W., Leahy D. A., 2008, ApJ, 677, 292

\bibitem[Tong et al.(2011)]{Tong11}
Tong H., Song L. M., Xu R. X. 2011, ApJ, 738, 31

\bibitem[Tong \& Xu(2011)]{Tong}
Tong H., Xu R. X., 2011, Int. Jour. Mod. Phys. E, 20, 15

\bibitem[Torii et al.(1998)]{Torii}
Torii K., Kinugasa K., Katayama K., Tsunemi H., Yamauchi S., 1998, ApJ, 503, 843

\bibitem[Tr\"{u}mper(2010)]{Trumper}
Tr\"{u}mper J. E., Zezas A., Ertan \"{u}., Kylafis N. D., 2010, A\&A, 518, 46

\bibitem[van der Meer et al.(2007)]{lmc}
van der Meer A., Kaper L., van Kerkwijk M. H., Heemskerk M. H. M., van den Heuvel E. P. J., 2007, A\&A, 473, 523

\bibitem[van Kerkwijk et al.(2004)]{kerk04}
van Kerkwijk M. H., Bassa C. G., Jacoby B. A., Jonker P. G. 2005, in Rasio F. A., Stairs I. H. eds., Binary Radio Pulsars, ASP Conference Series, 328, 357

\bibitem[Vasisht \& Gotthelf(1997)]{vas97}
Vasisht G., Gotthelf E. V., 1997, ApJ, 486, 129

%\bibitem[Vink \& Kuiper(2006)]{vink06}
%Vink  J., Kuiper L., 2006, MNRAS, 370, 14

\bibitem[Vrba et al.(2000)]{vrb00}
Vrba F. J., Henden A. A., Luginbuhl C. B., et al. 2000, ApJ, 533, 17

\bibitem[Wang et al.(2006)]{Wang}
Wang Z., Chakrabarty D., Kaplan D. L., 2006, Nature, 440, 772

\bibitem[Watts \& Strohmayer(2006)]{watts06}
Watts A. L., Strohmayer T. E., 2006, ApJ, 637, 117

\bibitem[Woods et al.(1999)]{Woods99}
Woods P. M., Kouveliotou C., van Paradijs J., Hurley K., Kippen R. M., Finger M. H., Briggs M. S., Dieters S., Fishman G. J., 1999, ApJ, 519, 139

\bibitem[Xu(2002)]{Xu02}
Xu R. X., 2002, ApJ, 570, L65

\bibitem[Xu et al.(2006)]{Xu}
Xu R. X., Tao D. J., Yang Y., 2006, MNRAS, 373, 85

\bibitem[Xu(2007)]{Xu07}
Xu R. X., 2007, Advances in space research, 40, 1453

\bibitem[Xu(2008)]{Xu08}
Xu R. X., 2008, Modern Physics Letters A, 23, 1629

\bibitem[Zanotti et al.(2011)]{za}
Zanotti O., Morozova V., Ahmedov B., 2011, arXiv:1111.0563

\end{thebibliography}

%%%%%%%%%%%%%%%%%%%%%%%%%%%%%%%%%%%%%%%
\end{document}
